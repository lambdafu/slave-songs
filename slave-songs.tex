% Copyright (C) 2007  Marcus Brinkmann <marcus@gnu.org>
%
% This song book is free software; you can redistribute it and/or
% modify it under the terms of the Creative Commons Legal Code
% Attribution-ShareALike as published by Creative Commons; either
% version 2.0 of the License, or (at your option) any later version.
%
% This song book is distributed in the hope that it will be useful,
% but WITHOUT ANY WARRANTY; without even the implied warranty of
% MERCHANTABILITY or FITNESS FOR A PARTICULAR PURPOSE.  See the
% Creative Commons Legal Code Attribution-ShareALike for more details.
%
% You should have received a copy of the Creative Commons Legal Code
% Attribution-ShareALike along with this score sheet; if not, write to
% Creative Commons, 543 Howard Street, 5th Floor,
% San Francisco, CA 94105-3013  United States

\documentclass[a5paper,10pt]{book}
% Lilypond uses New Century Schoolbook for lyrics.
\usepackage{newcent}
%% FIXME: We use footnotes, ``numbered'' per page.  Not symbolic,
%% because we can't get those into the lilypond score sheet without
%% the disfunctional TeX backend.
%%\usepackage[perpage,symbol*]{footmisc}
\usepackage[perpage]{footmisc}
\usepackage{chngpage}
\usepackage{multicol}
%% \usepackage{movie15}
%% \usepackage{hyperref}
%% \usepackage{graphicx}

%% FIXME: It would be a good idea to have some padding on a word that
%% carries a footnote marker in lyrics.
%% Uppercase section (and part) headings.

\begin{document}
%% Some gobal style settings.
%% We don't really want to use chapters.
\renewcommand \thesection {\arabic{section}.}

%% We need this for the title page.
%% FIXME: Doesn't work with dvips (ps and pdf are missing the font).
%% NOTE: Can be fixed by installing the Type1 fonts for ygoth.
%% NOTE: Let's not worry about this.
%%\newfont{\go}{ygoth.tfm scaled 1200}
\newcommand \go {\large}


%% Title page.

\title{{\Huge SLAVE SONGS}\\
\bigskip
{\small OF THE}\\
\bigskip
{\huge UNITED STATES.}}

\author{{\go New York:}\\A. SIMPSON \& CO.,}

%\date{\oldstylenums{1867}.}
\date{1867.}

\maketitle

\vfill

Digital Edition Version 1.0 ALPHA

Copyright \copyright\ 2007 by Marcus Brinkmann

This song book is free software; you can redistribute it and/or modify
it under the terms of the Creative Commons Legal Code
Attribution-ShareALike as published by Creative Commons; either
version 2.0 of the License, or (at your option) any later version.

This song book is distributed in the hope that it will be useful, but
WITHOUT ANY WARRANTY; without even the implied warranty of
MERCHANTABILITY or FITNESS FOR A PARTICULAR PURPOSE.  See the Creative
Commons Legal Code Attribution-ShareALike for more details.

You should have received a copy of the Creative Commons Legal Code
Attribution-ShareALike along with this score sheet; if not, write to
Creative Commons, 543 Howard Street, 5th Floor, San Francisco, CA
94105-3013 United States.

\tableofcontents

\chapter{Introduction.}

\pagenumbering{roman}


\textsc{The} musical capacity of the negro race has been recognized
for so many years that it is hard to explain why no systematic effort
has hitherto been made to collect and preserve their melodies.  More
than thirty years ago those plantation songs made their appearance
which were so extraordinarily popular for a while; and if ``Coal-black
Rose,'' ``Zip Coon'' and ``Ole Virginny nebber tire'' have been
succeeded by spurious imitations, manufactured to suit the somewhat
sentimental taste of our community, the fact that these were called
``negro melodies'' was itself a tribute to the musical genius of the
race.\footnote{It is not generally known that the beautiful air ``Long
time ago,'' or ``Near the lake where drooped the willow,'' was
borrowed from the negroes, by whom it was sung to words beginning,
``Way down in Raccoon Hollow.''}

The public had well-nigh forgotten these genuine slave songs, and with
them the creative power from which they sprung, when a fresh interest
was excited through the educational mission to the Port Royal islands,
in 1861.  The agents of this mission were not long in discovering the
rich vein of music that existed in these half-barbarous people, and
when visitors from the North were on the islands, there was nothing
that seemed better worth their while than to see a ``shout'' or hear
the ``people'' sing their ``sperichils.''  A few of these last, of
special merit,\footnote{The first seven spirituals in this collection,
which were regularly sung at the church.} soon became established
favorites among the whites, and hardly a Sunday passed at the church
on St.~Helena without ``Gabriel's Trumpet,'' ``I hear from Heaven
to-day,'' or ``Jehovah Hallelujah.''  The last time I myself heard
these was at the Fourth of July celebration, at the church, in 1864.
All of them were sung, and then the glorious shout, ``I can't stay
behind, my Lord,'' was struck up, and sung by the entire multitude
with a zest and spirit, a swaying of the bodies and nodding of the
heads and lighting of the countenances and rhythmical movement of the
hands, which I think no one present will ever forget.

Attention was, I believe, first publicly directed to these songs in a
letter from Miss McKim, of Philadelphia, to \emph{Dwight's Journal of
Music}, Nov.~8, 1862, from which some extracts will presently be
given.  At about the same time, Miss McKim arranged and published two
of them, ``Roll, Jordan'' (No.~1) and ``Poor Rosy'' (No.~8)---probably
on all accounts the two best specimens that could be selected.
Mr.~H.~G.~Spaulding not long after gave some well-chosen specimens of
the music in an article entitled ``Under the Palmetto,'' in the
\emph{Continental Monthly} for August, 1863, among them, ``O Lord,
remember me'' (No.~15), and ``The Lonesome Valley'' (No.~7).  Many
other persons interested themselves in the collection of words and
tunes, and it seems time at last that the partial collections in the
possession of the editors, and known by them to be in the possession
of others, should not be forgotten and lost, but that these relics of
a state of society which has passed away should be preserved while it
is still possible.\footnote{Only this last spring a valuable
collection of songs made at Richmond, Va., was lost in the
\emph{Wagner}.  No copy had been made from the original manuscript, so
that the labor of their collection was lost.  We had hoped to have the
use of them in preparing the present work.}

The greater part of the music here presented has been taken down by
the editors from the lips of the colored people themselves; when we
have obtained it from other sources, we have given credit in the table
of contents.  The largest and most accurate single collection in
existence is probably that made by Mr.~Charles P.~Ware, chiefly at
Coffin's Point, St.~Helena Island.  We have thought it best to give
this collection in its entirety, as the basis of the present work; it
includes all the hymns as far as No.~43.  Those which follow, as far
as No.~55, were collected by myself on the Capt.~John Fripp and
neighboring plantations, on the same island.  In all cases we have
added words from other sources and other localities, when they could
be obtained, as well as variations of the tunes wherever they were of
sufficient importance to warrant it.  Of the other hymns and songs we
have given the locality whenever it could be ascertained.

The difficulty experienced in attaining absolute correctness is
greater than might be supposed by those who have never tried the
experiment, and we are far from claiming that we have made no
mistakes.  I have never felt quite sure of my notation without a fresh
comparison with the singing, and have then often found that I had made
some errors.  I feel confident, however, that there are no mistakes of
importance.  What may appear to some to be an incorrect rendering, is
very likely to be a variation; for these variations are endless, and
very entertaining and instructive.

Neither should any one be repelled by any difficulty in adapting the
words to the tunes.  The negroes keep exquisite time in singing, and
do not suffer themselves to be daunted by any obstacle in the words.
The most obstinate Scripture phrases or snatches from hymns they will
force to do duty with any tune they please, and will dash heroically
through a trochaic tune at the head of a column of iambs with
wonderful skill.  We have in all cases arranged one set of words
carefully to each melody; for the rest, one must make them fit the
best he can, as the negroes themselves do.

The best that we can do, however, with paper and types, or even with
voices, will convey but a faint shadow of the original.  The voices of
the colored people have a peculiar quality that nothing can imitate;
and the intonations and delicate variations of even one singer cannot
be reproduced on paper.  And I despair of conveying any notion of the
effect of a number singing together, especially in a complicated
shout, like ``I can't stay behind, my Lord'' (No.~8), or ``Turn,
sinner, turn O!'' (No.~48).  There is no singing in
\emph{parts,}\footnote{``The high voices, all in unison, and the
admirable time and true accent with which their responses are made,
always make me wish that some great musical composer could hear these
semi-savage performances.  With a very little skilful adaptation and
instrumentation, I think one or two barbaric chants and choruses might
be evoked from them that would make the fortune of an
opera.''---\emph{Mrs.~Kemble's ``Life on a Georgian Plantation,''
p.~218.}} as we understand it, and yet no two appear to be singing the
same thing---the leading singer starts the words of each verse, often
improvising, and the others, who ``base'' him, as it is called, strike
in with the refrain, or even join in the solo, when the words are
familiar.  When the ``base'' begins, the leader often stops, leaving
the rest of his words to be guessed at, or it may be they are taken up
by one of the other singers.  And the ``basers'' themselves seem to
follow their own whims, beginning when they please and leaving off
when they please, striking an octave above or below (in case they have
pitched the tune too low or too high), or hitting some other note that
chords, so as to produce the effect of a marvellous complication and
variety, and yet with the most perfect time, and rarely with any
discord.  And what makes it all the harder to unravel a thread of
melody out of this strange network is that, like birds, they seem not
infrequently to strike sounds that cannot be precisely represented by
the gamut, and abound in ``slides from one note to another, and turns
and cadences not in articulated notes.''  ``It is difficult,'' writes
Miss~McKim, ``to express the entire character of these negro ballads
by mere musical notes and signs.  The odd turns made in the throat,
and the curious rhythmic effect produced by single voices chiming in
at different irregular intervals, seem almost as impossible to place
on the score as the singing of birds or the tones of an \AE{}olian
Harp.''  There are also apparent irregularities in the time, which it
is no less difficult to express accurately, and of which Nos.~10, 130,
131, and (eminently) 128, are examples.

Still, the chief part of the negro music is \emph{civilized} in its
character---partly composed under the influence of association with
the whites, partly actually imitated from their music.  In the main it
appears to be original in the best sense of the word, and the more we
examine the subject, the more genuine it appears to us to be.  In a
very few songs, as Nos.~19, 23, and 25, strains of familiar tunes are
readily traced; and it may easily be that others contain strains of
less familiar music, which the slaves heard their masters sing or
play.\footnote{We have rejected as spurious ``Give me Jesus,'' ``Climb
Jacob's Ladder,'' (both sung at Port Royal), and ``I'll take the wings
of the morning,'' which we find in Methodist hymn-books.  A few
others, the character of which seemed somewhat suspicious, we have not
%% CHANGED: I appended a full stop.
felt at liberty to reject without direct evidence.}

On the other hand there are very few which are of an intrinsically
barbaric character, and where this character does appear, it is
chiefly in short passages, intermingled with others of a different
character.  Such passages may be found perhaps in Nos.~10, 12, and 18;
and ``Becky Lawton,'' for instance (No.~29), ``Shall I die?'' (No.~52)
``Round the corn, Sally'' (No.~87), and ``O'er the crossing'' (No.~93)
may very well be purely African in origin.  Indeed, it is very likely
that if we had found it possible to get at more of their secular
music, we should have come to another conclusion as to the proportion
of the barbaric element.  A gentleman in Delaware writes:

``We must look among their non-religious songs for the purest
specimens of negro minstrelsy, It is remarkable that they have
themselves transferred the best of these to the uses of their
churches---I suppose on Mr.~Wesley's principle that `it is not right
the Devil should have all the good tunes.'  Their leaders and
preachers have not found this change difficult to effect; or at least
they have taken so little pains about it that one often detects the
profane \emph{cropping out,} and revealing the origin of their most
solemn `hymns,' in spite of the best intentions of the poet and
artist.  Some of the best \emph{pure negro} songs I have ever heard
were those that used to be sung by the black stevedores, or perhaps
the crews themselves, of the West India vessels, loading and unloading
at the wharves in Philadelphia and Baltimore.  I have stood for more
than an hour, often, listening to them, as they hoisted and lowered
the hogsheads and boxes of their cargoes; one man taking the burden of
the song (and the slack of the rope) and the others striking in with
the chorus.  They would sing in this way more than a dozen different
songs in an hour; most of which might indeed be warranted to contain
`nothing religious'---a few of them, `on the contrary, quite the
reverse'---but generally rather innocent and proper in their language,
and strangely attractive in their music; and with a volume of voice
that reached a square or two away.  That plan of labor has now passed
away, in Philadelphia at least, and the songs, I suppose, with it.  So
that these performances are to be heard only among black sailors on
%% CHANGED: I inserted a hyphen between ``of'' and ``the''.
their vessels, or 'long-shore men in out-of-the-way place, where
opportunities for respectable persons to hear them are rather few.''

These are the songs that are still heard upon the Mississippi
steamboats---wild and strangely fascinating---one of which we have
been so fortunate as to secure for this collection.  This, too, is no
doubt the music of the colored firemen of Savannah, graphically
described by Mr.~Kane~O'Donnel, in a letter to the Philadelphia
\emph{Press,} and one of which he was able to contribute for our use.
Mr.~E.~S.~Philbrick was struck with the resemblance of some of the
rowing tunes at Port-Royal to the boatmen's songs he had heard upon
the Nile.

The greater number of the songs which have come into our possession
seem to be the natural and original production of a race of remarkable
musical capacity and very teachable, which has been long enough
associated with the more cultivated race to have become imbued with
the mode and spirit of European music---often, nevertheless, retaining
a distinct tinge of their native Africa.

The words are, of course, in a large measure taken from Scripture, and
from the hymns heard at church; and for this reason these religious
songs do not by any means illustrate the full extent of the debasement
of the dialect.  Such expressions as ``Cross Jordan,'' ``O Lord,
remember me,'' ``I'm going home,'' ``There's room enough in Heaven for
you,'' we find abundantly in Methodist hymn-books; but with much
searching I have been able to find hardly a trace of the tunes.  The
words of the fine hymn, ``Praise, member'' (No.~5), are found, with
very little variation, in ``Choral Hymns'' (No.~138).  The editor of
this collection informs us, however, that many of his songs were
learned from negroes in Philadelphia, and Lt.-Col.~Trowbridge tells us
that he heard this hymn, before the war, among the colored people of
Brooklyn.\footnote{We have generally preserved the words as sung, even
where clearly nonsensical, as in No.~89; so ``Why don't you move so
slow?'' (No.~22).  We will add that ``Paul and Silas, bound in jail''
(No.~4) is often sung ``Bounden Cyrus born in jail,'' and the words of
No.~11 would appear as ``I take my tex in Matchew and by de
Revolutions---I know you by your gammon,'' \&c.; so ``Ringy Rosy Land''
for ``Ring Jerusalem.''}  For some very comical specimens of the way
in which half-understood words and phrases are distorted by them, see
Nos.~22, 23.  Another illustration is given by
Col.~Higginson:\footnote{\emph{Atlantic Monthly,} June 1867.}

``The popular camp-song of `Marching Along' was entirely new to them
until our quartermaster taught it to them at my request.  The words
`Gird on the armor' were to them a stumbling-block, and no wonder,
until some ingenious ear substituted `Guide on de army,' which was at
once accepted and became universal.  `We'll guide on de army, and be
marching along,' is now the established version on the Sea Islands.''

I never fairly heard a secular song among the Port Royal freedmen, and
never saw a musical instrument among them.  The last violin, owned by
a ``worldly man,'' disappeared from Coffin's Point ``de year gun shoot
at Bay Pint.''\footnote{\emph{i.~e.}, November, 1861, when Hilton Head
was taken by Admiral Dupont---a great date on the islands.}  In other
parts of the South, ``fiddle-sings,'' ``devil-songs,'' ``corn-songs,''
``jig-tunes,'' and what not, are common; all the world knows the
banjo, and the ``Jim Crow'' songs of thirty years ago.  We have
succeeded in obtaining only a very few songs of this character.  Our
intercourse with the colored people has been chiefly through the work
of the Freedmen's Commission, which deals with the serious and earnest
side of the negro character.  It is often, indeed, no easy matter to
persuade them to sing their old songs, even as a curiosity, such is
the sense of dignity that has come with freedom.  It is earnestly to
be desired that some person, who has the opportunity, should make a
collection of these now, before it is too late.

In making the present collection, we have only gleaned upon the
surface, and in a very narrow field.  The wealth of material still
awaiting the collector can be guessed from a glance at the localities
of those we have, and from the fact, mentioned above, that of the
first forty-three of the collection most were sung upon a single
plantation, and that it is very certain that the stores of this
plantation were by no means exhausted.  Of course there was constant
intercourse between neighboring plantations; also between different
States, by the sale of slaves from one to another.  But it is
surprising how little this seems to have affected local songs, which
are different even upon adjoining plantations.  The favorite of them
all, ``Roll, Jordan'' (No.~1), is sung in Florida, but not, I believe,
in North Carolina.  ``Gabriel's Trumpet'' (No.~4) and ``Wrestle on,
%% CHANGED: Added a period to No.
Jacob'' (No.~6) probably came from Virginia, where they are sung
without much variation from the form usual at Port Royal; No.~6 is
also sung in Maryland.\footnote{It is worthy of notice that a song
much resembling ``Poor Rosy'' was heard last Spring from the boat
hands of an Ohio River steamboat---the only words caught being ``Poor
Molly, poor gal.''}  ``John, John of the Holy Order'' (No.~22) is
traced in Georgia and North Carolina, and ``O'er the Crossing''
(No.~93) appears to be the Virginia original, variations of which are
found in South Carolina, Georgia, and Tennessee.  As illustrations of
the slowness with which these songs travel, it may be mentioned that
the ``Graveyard'' (No.~21), which was frequently sung on
Capt.~John~Fripp's plantation in the winter of 1863--4, did not reach
Coffin's Point (five miles distant) until the following Spring.  I
heard it myself at Pine Grove, two miles from the latter place, in
March.  Somewhere upon this journey this tune was strikingly altered,
as will be seen from the variation given, which is the form in which I
was accustomed to hear it.  Nos.~38, 41, 42, 43, 118, 119, 122, 123,
were brought to Coffin's Point after Mr.~Ware left, by refugees
returning to the plantation from ``town'' and the Main.  No.~74,
likewise, ``Nobody knows the trouble I see,'' which was common in
Charleston in 1865, has since been carried to Coffin's Point, very
little altered.

These hymns will be found peculiarly interesting in illustrating the
feelings, opinions and habits of the slaves.  Of the dialect I shall
presently speak at some length.  One of their customs, often alluded
to in the songs (as in No.~19), is that of wandering through the woods
and swamps, when under religious excitement, like the ancient
bacchantes.  To get religion is with them to ``fin' dat ting.''  Molsy
described thus her sister's experience in searching for religion:
``Couldn't fin' dat leetle ting---hunt for 'em---huntin' for 'em all
de time---las' foun' 'em.''  And one day, on our way to see a
``shout,'' we asked Bristol whether he was going:---''No, ma'am,
wouldn't let me in---hain't foun' dat ting yet---hain't been on my
knees in de swamp.''  Of technical religious expressions, ``seeker,''
``believer,'' ``member,'' \&c., the songs are full.

The most peculiar and interesting of their customs is the ``shout,''
an excellent description of which we are permitted to copy from the
N.~Y.~\emph{Nation} of May~30, 1867:

``This is a ceremony which the white clergymen are inclined to
discountenance, and even of the colored elders some of the more
discreet try sometimes to put on a face of discouragement; and
although, if pressed for Biblical warrant for the shout, they
generally seem to think `he in de Book,' or `he dere-da in Matchew,'
still it is not considered blasphemous or improper if `de chillen' and
`dem young gal' carry it on in the evening for amusement's sake, and
with no well-defined intention of `praise.'  But the true `shout'
takes place on Sundays or on `praise'-nights through the week, and
either in the praise-house or in some cabin in which a regular
religious meeting has been held.  Very likely more than half the
population of the plantation is gathered together.  Let it be the
evening, and a light-wood fire burns red before the door of the house
and on the hearth.  For some time one can hear, though at a good
distance, the vociferous exhortation or prayer of the presiding elder
or of the brother who has a gift that way, and who is not `on the back
seat,'---a phrase, the interpretation of which is, `under the censure
of the church authorities for bad behavior;'---and at regular
intervals one hears the elder `deaconing' a hymn-book hymn, which is
sung two lines at a time, and whose wailing cadences, borne on the
night air, are indescribably melancholy.  But the benches are pushed
back to the wall when the formal meeting is over, and old and young,
men and women, sprucely-dressed young men, grotesquely half-clad
field-hands---the women generally with gay handkerchiefs twisted about
their heads and with short skirts---boys with tattered shirts and
men's trousers, young girls barefooted, all stand up in the middle of
the floor, and when the `sperichil' is struck up, begin first walking
and by-and-by shuffling round, one after the other, in a ring.  The
foot is hardly taken from the floor, and the progression is mainly due
to a jerking, hitching motion, which agitates the entire shouter, and
soon brings out streams of perspiration.  Sometimes they dance
silently, sometimes as they shuffle they sing the chorus of the
spiritual, and sometimes the song itself is also sung by the dancers.
But more frequently a band, composed of some of the best singers and
of tired shouters, stand at the side of the room to `base' the others,
singing the body of the song and clapping their hands together or on
the knees.  Song and dance are alike extremely energetic, and often,
when the shout lasts into the middle of the night, the monotonous
thud, thud of the feet prevents sleep within half a mile of the
praise-house.''

In the form here described, the ``shout'' is probably confined to
South Carolina and the States south of it.  It appears to be found in
Florida, but not in North Carolina or Virginia.  It is, however, an
interesting fact that the term ``shouting'' is used in Virginia in
reference to a peculiar motion of the body not wholly unlike the
Carolina shouting.  It is not unlikely that this remarkable religious
ceremony is a relic of some native African dance, as the Romaika is of
the classical Pyrrhic.  Dancing in the usual way is regarded with
great horror by the people of Port Royal, but they enter with infinite
zest into the movements of the ``shout.''  It has its connoisseurs,
too.  ``Jimmy great shouter,'' I was told; and Jimmy himself remarked
to me, as he looked patronizingly on a ring of young people, ``Dese
yere worry deyseff---we don't worry weseff.''  And indeed, although
the perspiration streamed copiously down his shiny face, he shuffled
round the circle with great ease and grace.

The shouting may be to any tune, and perhaps all the Port Royal hymns
here given are occasionally used for this purpose; so that our cook's
classification into ``sperichils'' and ``runnin' sperichils''
(shouts), or the designation of certain ones as sung ``just sittin'
round, you know,'' will hardly hold in strictness.  In practice,
however, a distinction is generally observed.  The first seven, for
instance, favorite hymns in the St. Helena church, would rarely, if
ever, be used for shouting; while probably on each plantation there is
a special set in common use.  On my plantation I oftenest heard ``Pray
all de member'' (No.~47), ``Bell da ring'' (No.~46), ``Shall I die?''
(No.~52), and ``I can't stay behind, my Lord'' (No.~8).  The shouting
step varied with the tune; one could hardly dance with the same spirit
to ``Turn, sinner,'' or ``My body rock 'long fever,'' as to ``Rock o'
Jubilee,'' or ``O Jerusalem, early in de morning.''  So far as I can
learn, the shouting is confined to the Baptists; and it is, no doubt,
to the overwhelming preponderance of this denomination on the Sea
Islands that we owe the peculiar richness and originality of the music
%% CHANGED: I appended a full stop.
there.

The same songs are used for rowing as for shouting.  I know only one
pure boat-song, the fine lyric, ``Michael row the boat ashore''
(No.~31); and this I have no doubt is a real spiritual---it being the
archangel Michael that is addressed.  Among the most common rowing
tunes were Nos.~5, 14, 17, 27, 28, 29, 30, 31, 32, 33, 36, 46.  ``As I
have written these tunes,'' says Mr.~Ware, ``two measures are to be
sung to each stroke, the first measure being accented by the beginning
of the stroke, the second by the rattle of the oars in the row-locks.
On the passenger boat at the [Beaufort] ferry, they rowed from sixteen
to thirty strokes a minute; twenty-four was the average.  Of the tunes
I have heard, I should say that the most lively were `Heaven bell
a-ring' (No.~27), `Jine 'em' (No.~28), `Rain fall' (No.~29), `No man'
(No.~14), `Bell da ring' (No.~46), and `Can't stay behind;' and that
`Lay this body down' (No.~26), `Religion so sweet' (No.~17), and
`Michael row' (No.~31), were used when the load was heavy or the tide
was against us.  I think that the long hold on `Oh,' in `Rain fall,'
was only used in rowing.  When used as a `shout' I am quite sure that
it occupied only one measure, as in the last part of the verse.  One
noticeable thing about their boat-songs was that they seemed often to
be sung just a trifle behind time; in `Rain fall,' for instance,
`Believer cry holy' would seem to occupy more than its share of the
stroke, the `holy' being prolonged till the very beginning of the next
stroke; indeed, I think Jerry often hung on his oar a little just
there before dipping it again.''\footnote{For another curious
circumstance in rowing, see note to ``Rain fall,'' No.~29.}

As to the composition of these songs, ``I always wondered,'' says
Col.~Higginson, ``whether they had always a conscious and definite
origin in some leading mind, or whether they grew by gradual
accretion, in an almost unconscious way.''  On this point I could get
no information, though I asked many questions, until at last, one day
when I was being rowed across from Beaufort to Ladies' Island, I found
myself, with delight, on the actual trail of a song.  One of the
oarsmen, a brisk young fellow, not a soldier, on being asked for his
theory of the matter, dropped out a coy confession.  `Some good
sperituals,' he said, `are start jess out o' curiosity.  I been
a-raise a sing, myself, once.'

``My dream was fulfilled, and I had traced out, not the poem alone,
but the poet.  I implored him to proceed.

``{}`Once we boys,' he said, `went for tote some rice, and de
nigger-driver, he keep a-callin' on us; and I say, `O, de ole
nigger-driver!'  Den anudder said, `Fust ting my mammy told me was,
notin' so bad as nigger-drivers.'  Den I made a sing, just puttin' a
word, and den anudder word.'

``Then he began singing, and the men, after listening a moment, joined
in the chorus as if it were an old acquaintance, though they evidently
had never heard it before.  I saw how easily a new `sing' took root
among them.''

A not inconsistent explanation is that given on page~12 of an
``Address delivered by J.~Miller McKim, in Sansom Hall, Philadelphia,
July~9, 1862.''

``I asked one of these blacks---one of the most intelligent of them
[Prince Rivers, Sergeant 1st Reg.~S.~C.~V.]---where they got these
songs.  `Dey make 'em, sah.'  `How do they make them?'  After a pause,
evidently casting about for an explanation, he said: `I'll tell you,
it's dis way.  My master call me up, and order me a short peck of corn
and a hundred lash.  My friends see it, and is sorry for me.  When dey
come to de praise-meeting dat night dey sing about it.  Some's very
good singers and know how; and dey work it in---work it in, you know,
till they get it right; and dat's de way.'  A very satisfactory
explanation; at least so it seemed to me.''

We were not so fortunate as Col.~Higginson in our search for a poet.
Cuffee at Pine Grove did, to be sure, confess himself the author of
``Climb Jacob's Ladder;''---unfortunately, we afterwards found it in a
Northern hymn book.  And if you try to trace out a new song, and ask,
``Where did you hear that?'' the answer will be, ``One strange man
come from Eding's las' praise-night and sing 'em in praise-house, and
de people catch 'em;'' or ``Titty 'Mitta [sister Amaritta] fetch 'em
from Polawana, where she tuk her walk gone spend Sunday.  Some of her
fahmly sing 'em yonder.''  ``But what does `Ringy rosy land' [Ring
Jerusalem, No.~21] mean?'' ``Me dunno.''

Our title, ``Slave Songs,'' was selected because it best described the
contents of the book.  A few of those here given (Nos.~64, 59) were,
to be sure, composed since the proclamation of emancipation, but even
these were inspired by slavery.  ``All, indeed, are valuable as an
expression of the character and life of the race which is playing such
a conspicuous part in our history.  The wild, sad strains tell, as the
sufferers themselves could, of crushed hopes, keen sorrow, and a dull,
daily misery, which covered them as hopelessly as the fog from the
rice swamps.  On the other hand, the words breathe a trusting faith in
rest for the future---in `Canaan's air and happy land,' to which their
eyes seem constantly turned.''

Our original plan hardly contemplated more than the publication of the
Port Royal spirituals, some sixty in all, which we had supposed we
could obtain, with perhaps a few others in an appendix.  As new
materials came into our hands, we enlarged our plan to the present
dimensions.  Next to South Carolina, we have the largest number from
Virginia; from the other States comparatively few.  Few as they are,
however, they appear to indicate a very distinct character in
different States.  Contrary to what might be expected, the songs from
Virginia are the most wild and strange.  ``O'er the Crossing,''
(No.~93) is peculiarly so; but ``Sabbath has no end'' (No.~89),
``Hypocrite and Concubine'' (No.~91), ``O shout away'' (No.~92), and
``Let God's saints come in'' (No.~99), are all distinguished by odd
intervals and a frequent use of chromatics.  The songs from North
Carolina are also very peculiar, although in a different way, and make
one wish for more specimens from that region.  Those from Tennessee
and Florida are most like the music of the whites.

We had hoped to obtain enough secular songs to make a division by
themselves; there are, however, so few of these that it has been
decided to intersperse them with the spirituals under their respective
States.  They are highly characteristic, and will be found not the
least interesting of the contents of this work.

It is, we repeat, already becoming difficult to obtain these songs.
Even the ``spirituals'' are going out of use on the plantations,
superseded by the new style of religious music, ``closely imitated
from the white people, which is solemn, dull and nasal, consisting in
repeating two lines of a hymn and then singing it, and then two more,
\emph{ad infinitum}.  They use for this sort of worship that one
everlasting melody, which may be remembered by all persons familiar
with Western and Southern camp-meetings, as applying equally well to
long, short or common metre.  This style of proceeding they evidently
consider the more dignified style of the two, as being a closer
imitation of white, genteel worship---having in it about as little
soul as most stereotyped religious forms of well instructed
congregations.''\footnote{Mrs.~H.~B.~Stowe, in \emph{Watchman and
Reflector,} April 1867.}

It remains to speak of points connected with the typography of the
songs.

We have aimed to give all the characteristic variations which have
come into our hands, whether as single notes or whole lines, or even
longer passages; and of words as well as tunes.  Many of these will be
found very interesting and instructive.  The variations in words are
given as foot-notes---the word or group of words in the note, to be
generally substituted for that which precedes the mark: and it may be
observed, although it seems hardly necessary, that these variations
are endless; such words as ``member,'' ``believer,'' ``seeker,'' and
all names, male and female, may be brought in wherever appropriate.
%% EDITED: I apppended a semi-colon.
We have not always given all the sets of words that we have received;
often they are improvised to such an extent that this would be almost
impracticable.  In Nos.~16, 17, 19, etc., we have given them very
copiously, for illustration; in others we have omitted the least
interesting ones.  In spelling, we proposed to ourselves the rule well
stated by Col.~Higginson at the commencement of his collection: ``The
words will be here given, as nearly as possible, in the original
dialect; and if the spelling seems sometimes inconsistent, or the
misspelling insufficient, it is because I could get no nearer.''

As the negroes have no part-singing, we have thought it best to print
only the melody; what appears in some places as harmony is really
variations in single notes.  And, in general, a succession of such
notes turned in the same direction indicates a single longer
variation.  Words in a parenthesis, with small notes, (as ``Brudder
Sammy'' in No.~21), are interjaculatory; it has not, however, been
possible to maintain entire consistency in this matter.  Sometimes, as
``no man'' and ``O no man,'' in No.~14, interchangeable forms are put,
for convenience sake, in different parts of the tune.

It may sometimes be a little difficult, for instance in Nos.~9, 10, 20
and 27, to determine precisely which part of the tune each verse
belongs to; in these cases we have endeavored to indicate it as
clearly as is in our power.  However much latitude the reader may take
in all such matters, he will hardly take more than the negroes
themselves do.  In repeating, it may be observed that the custom at
Port Royal is to repeat the first part of the tune over and over, it
may be a dozen times, before passing to the ``turn,'' and then to do
the same with that.  In the Virginia songs, on the other hand, the
chorus is usually sung twice after each verse---often the second time
with some such interjaculatory expression as ``I say now,'' ``God say
you must,'' as given in No.~99.

We had some thought of indicating with each the \emph{tempo} of the
different songs, but have concluded to print special directions for
singing by themselves.  It should be remarked, however, that the same
tune varied in quickness on different occasions.  ``As the same
songs,'' writes Miss McKim, ``are sung at every sort of work, of
course the \emph{tempo} is not always alike.  On the water, the oars
dip `Poor Rosy' to an even \emph{andante;} a stout boy and girl at the
hominy mill will make the same `Poor Rosy' fly, to keep up with the
whirling stone; and in the evening, after the day's work is done,
`Heab'n shall-a be my home' peals up slowly and mournfully from the
distant quarters.  One woman, a respectable house-servant, who had
lost all but one of her twenty-two children, said to me: `Pshaw! don't
har to dese yer chil'en, missee.  Dey just rattles it off---dey don't
know how for sing it.  I likes `Poor Rosy' better dan all de songs,
but it can't be sung widout \emph{a full heart and a troubled
%% CHANGED: I appended the closing single and double quote.
sperrit}.'{}''

The rests, by the way, do not indicate a cessation in the music, but
only in part of the singers.  They overlap in singing, as already
described, in such a degree that at no time is there any complete
pause.  In ``A House in Paradise'' (No. 40) this overlapping is most
marked.

\bigskip

\textsc{It} will be noticed that we have spoken chiefly of the negroes
of the Port Royal islands, where most of our observations were made,
and most of our materials collected.  The remarks upon the dialect
which follow have reference solely to these islands, and indeed almost
exclusively to a few plantations at the northern end of St.~Helena
Island.  They will, no doubt, apply in a greater or less degree to the
entire region of the southeasterly slave States, but not to other
portions of the South.  It should also be understood that the
corruptions and peculiarities here described are not universal, even
here.  There are all grades, from the rudest field-hands to mechanics
and house-servants, who speak with a considerable degree of
correctness, and perhaps few would be found so illiterate as to be
guilty of them all.

Ordinary negro talk, such as we find in books, has very little
resemblance to that of the negroes of Port Royal, who have been so
isolated heretofore that they have almost formed a dialect of their
own.  Indeed, the different plantations have their own peculiarities,
and adepts profess to be able to determine by the speech of a negro
what part of an island he belongs to, or even, in some cases, his
plantation.  I can myself vouch for the marked peculiarities of speech
of one plantation from which I had scholars, and which was hardly more
than a mile distant from another which lacked these peculiarities.
Songs, too, and, I suppose, customs, vary in the same way.

A stranger, upon first hearing these people talk, especially if there
is a group of them in animated conversation, can hardly understand
them better than if they spoke a foreign language, and might, indeed,
easily, suppose this to be the case.  The strange words and
pronunciations, and frequent abbreviations, disguise the familiar
features of one's native tongue, while the rhythmical modulations, so
characteristic of certain European languages, give it an utterly
un-English sound.  After six months' residence among them, there were
scholars in my school, among the most constant in attendance, whom I
could not understand at all, unless they happened to speak very
slowly.

With these people the process of ``phonetic decay'' appears to have
gone as far, perhaps, as is possible, and with it an extreme
simplification of etymology and syntax.  There is, of course, the
usual softening of \emph{th} and \emph{v,} or \emph{f,} into \emph{d}
and \emph{b;} likewise a frequent interchange of \emph{v} and
\emph{w,} as \emph{veeds} and \emph{vell} for \emph{weeds} and
\emph{well;} \emph{woices} and \emph{punkin wine,} for \emph{voices}
and \emph{pumpkin vine.}  ``De wile' (\emph{vilest}) sinner may
return'' (No.~48).  This last example illustrates also their constant
%% CHANGED: "bro,'" to "bro',".
habit of clipping words and syllables, as \emph{lee' bro',} for
\emph{little brother;} \emph{pl\"ant'shun,} for \emph{plantation.}
The lengthening of short vowels is illustrated in both these
(\emph{a,} for instance, rarely has its short English sound).  ``Een
(in) dat mornin' all day'' (No.~56).

Strange words are less numerous in their \emph{patois} than one would
suppose, and, few as they are, most of them maybe readily derived from
English words.  Besides the familiar \emph{buckra,} and a few proper
names, as Cuffy, Quash, and perhaps Cudjo, I only know of
\emph{churray} (spill), which may be ``throw 'way;'' \emph{oona} or
\emph{ona,} ``you'' (both singular and plural, and used only for
friends), as ``Ona build a house in Paradise'' (No.~40); and
\emph{aw,} a kind of expletive, equivalent to ``to be sure,'' as,
``Dat clot' cheap.''  ``Cheap aw.''  ``Dat Monday one lazy boy.''
``Lazy aw---I 'bleege to lick 'em.''

Corruptions are more abundant.  The most common of them are these:
\emph{Yearde} (hear), as in Nos.~3, etc.  ``Flora, did you see that
cat?''  ``No ma'am, but I yearde him holler.''  ``\emph{Sh'um,}'' a
corruption of \emph{see 'em,} applied (as \emph{'em} is) to all
genders and both numbers.  ``Wan' to see how Beefut (Beaufort)
stan'---nebber sh'um since my name Adam.''  \emph{Huddy} (how-do?),
pronounced \emph{how-dy} by purists, is the common term of greeting,
as in the song No.~20, ``Tell my Jesus huddy O.''  ``Bro' (brother)
Quash sen' heap o' howdy.''  \emph{Studdy,} (steady) is used to denote
any continued or customary action.  ``He studdy 'buse an' cuss we,''
was the complaint entered by some little children against a large
girl.  ``I studdy talk hard, but you no yearde me,'' was Rina's
defence when I reproved her for not speaking loud enough.  When we
left, we were told that we must ``studdy come back.''  Here, however,
it seems to mean \emph{steady.}  \emph{Titty} is used for mother or
oldest sister; thus, Titty Ann was the name by which the children of
our man-of-all work knew their mother, Ann.  \emph{Sic-a} or
\emph{sake-a,} possibly a condensation of \emph{same} and \emph{like.}
``Him an' me grow up sic-a brudder an' sister.''  \emph{Enty} is a
curious corruption, I suppose of \emph{ain't he,} used like our ``Is
that so?'' in reply to a statement that surprises one.  ``Robert, you
%% EDITED: I added an opening double quote before ``John''.
have n't written that very well.''  ``Enty, sir?''  ``John, it's going
to rain to-day.''  ``Enty, sir?''  \emph{Day-clean} is used for
\emph{day-break.}  ``Do, day-clean, for let me go see Miss Ha'yet; and
de day wouldn't clean.''  \emph{Sun-up} is also common.  \emph{Chu'}
for ``this'' or ``that there;'' as ``Wha' chu?''  ``See one knife
chu?''  \emph{Say} is used very often, especially in singing, as a
kind of expletive; ``(Say) when you get to heaven (say) you 'member
me.'' (No.~27.)  ``Ain't you know say cotton de-de?''  In the last
sentence ``de-de'' (accent on first syllable) means ``is
there;''---the first \emph{de,} a corruption of \emph{does} for
\emph{is,} will be explained presently; the other is a very common
form for \emph{dere,} there.

I do not remember any other peculiar words, but several words used
peculiarly.  \emph{Cuss} is used with great latitude, to denote any
offensive language.  ``Him cuss me 'git out.''  ``Ahvy (Abby) do cuss
me,'' was the serious-sounding, but trifling accusation made by a
little girl against her seat-mate.  \emph{Stan'} is a very common
word, in the sense of \emph{look.}  ``My back stan' like white man,''
was a boast which meant that it was not scarred with the lash.  ``Him
stan' splendid, ma'am,'' of the sitting of a dress.  I asked a group
of boys one day the color of the sky.  Nobody could tell me.
Presently the father of one of them came by, and I told him their
ignorance, repeating my question with the same result as before.  He
grinned: ``Tom, how sky stan'?''  ``Blue,'' promptly shouted Tom.
\emph{Both} they seldom use; generally ``all-two,'' or emphatically,
``all-two boff togedder.''  \emph{One} for \emph{alone.}  ``Me one,
and God,'' answered an old man in Charleston to the question whether
he escaped alone from his plantation.  ``Gone home one in de dark,''
for alone.  ``Heab'n 'nuff for me one'' (\emph{i. e.,} I suppose,
%% EDITED: Moved the full stop out of the parenthesis.
``for my part''), says one of their songs (No.~46).  \emph{Talk} is
one of their most common words, where we should use \emph{speak} or
\emph{mean.}  ``Talk me, sir?'' asks a boy who is not sure whether you
mean him or his comrade.  ``Talk lick, sir?  nuffin but lick,'' was
the answer when I asked whether a particular master used to whip his
slaves.  \emph{Call} is used to express relationship as, ``he call him
aunt.''  \emph{Draw,} for receiving in any way---derived from the
usage of drawing a specific amount of supplies at stated times.  ``Dey
draw letter,'' was the remark when a mail arrived and was distributed
among us whites.  \emph{Meet} is used in the sense of \emph{find.}
``I meet him here an' he remain wid me,'' was the cook's explanation
when a missing chair was found in the kitchen.  When I remarked upon
the absurdity of some agricultural process---''I meet 'em so an' my
fader meet 'em so,'' was the sufficient answer.  A grown man, laboring
over the mysteries of simple addition, explained the gigantic answer
he had got by ``I meet two row, and I set down two.''  ``I meet you
dere, sir,'' said Miller frankly, when convinced in an argument.  Too
\emph{much} is the common adverb for a high degree of a quality; ``he
bad \emph{too} much'' was the description of a hard master.
\emph{Gang,} for any large number; ``a whole gang of slate-pencils.''
\emph{Mash} in the sense of crush; ``mammy mash 'em,'' when the goat
had killed one of her kids by lying on it.  \emph{Sensibble} and
\emph{hab sense} are favorite expressions.  A scholar would ask me to
make him ``sensibble'' of a thing.  ``Nebber sh'um since I hab sense''
(\emph{i.~e.,} since I was old enough to know).  \emph{Stantion}
(substantial) was a favorite adjective at Coffin's Point.
\emph{Strain} is also a favorite word.  ``Dem boy strain me,''
explained Billy, when some younger boys were attempting to \emph{base}
him.  ``I don't want to give more nor fifty-five dollar for a horse,''
said Quash, ``but if dey strain you, you may give fifty-six.''  ``Dat
tune \emph{so} strainful,'' said Rose.

The letters \emph{n,} \emph{r} and \emph{y} are used euphonically.
``He de baddes' little gal from y'ere to n'Europe,'' said Bristol of
his troublesome niece Venus; |ought to put him on a bar'l, an' den he
fall 'sleep an' fall down an' hut heself, an' dat make him more
sensibble.''  ``He n'a comin', sir,'' was often said of a missing
scholar.  At first, I took the \emph{n} for a negative.  I set Gib one
day to picking out \emph{E'}s from a box of letters.  He could not
distinguish \emph{E} from \emph{F,} and at last, discouraged with his
repeated failures, explained, holding out an \emph{F,} ``dis y'ere
stan' sic-a-r-\emph{um.}''  (This looks like that.)  It is suggested
also that \emph{d} is used in the same way, in ``He d'a comin';'' and
\emph{s,} in singing, for instance, ``'Tis wells and good'' (No.~25).
So the vowel \emph{a;} ``De foxes have-a hole'' (No.~2), ``Heaven bell
a-ring'' (No.~27).

The most curious of all their linguistic peculiarities is perhaps the
following.  It is well known that the negroes in all parts of the
South speak of their elders as ``uncle'' and ``aunt,---''\footnote{In
South Carolina ``daddy'' and ``maum'' are more common.} from a feeling
of politeness, I do not doubt; it seemed disrespectful to use the bare
name, and from \emph{Mr.} and \emph{Mrs.} they were debarred.  On the
Sea Islands a similar feeling has led to the use of \emph{cousin}
towards their equals.  Abbreviating this, after their fashion, they
get \emph{co'n} or \emph{co'} (the vowel sound \emph{u} as in
\emph{cousin}) as the common title when they speak of one another; as,
C'Abram, Co' Robin, Co'n Emma, C'Isaac, Co' Bob.  \emph{Bro'}
(brother) and \emph{Si'} (sister) and even \emph{T'} (Titty) are also
often used in the same way; as, Bro' Paris, Si' Rachel, T' Jane.  A
friend insists that \emph{Cudjo} is nothing but Co' Joe.

\emph{Where} and \emph{when} are hardly used, at least by the common
class of negroes.  The question ``Where did you spill the milk?'' was
answered only with a stare; but ``which way milk churray?'' brought a
ready response.  ``What side you stayin', sir?'' was one of the first
questions put to me.  Luckily I had been initiated, and was able to
answer it correctly.

There is probably no speech that has less inflection, or indeed less
power of expressing grammatical relation in any way.  It is perhaps
not too strong to say that the field-hands make no distinction of
gender, case, number, tense, or voice.  The pronouns are to be sure
distinguished more or less by the more intelligent among them, and all
of these, unless perhaps \emph{us,} are occasionally heard.
\emph{She} is rare; \emph{her} still more so; \emph{him} being
commonly used for the third person singular of all cases and genders;
\emph{'em,} if my memory serves me rightly, only for the objective
case, but for all genders and both numbers.  \emph{He,} or \emph{'e,}
is, I should think, most common as possessive.  ``Him lick we'' might
mean a girl as well as a boy.  Thus \emph{we} is distinguished from
\emph{I} or \emph{me,} and \emph{dey} or \emph{dem} from \emph{him} or
\emph{dat;} and these are, I think, the only distinctions made in
number.  ``Dat cow,'' is singular, ``dem cow'' plural; ``Sandy hat''
would mean indifferently Sandy's hat or hats; ``nigger-house'' means
the collection of negro-houses, and is, I suppose, really a plural.

I do not know that I ever heard a real possessive case, but they have
begun to develop one of their own, which is a very curious
illustration of the way inflectional forms grow up.  If they wish to
make the fact of possession at all emphatic or distinct, they use the
word ``own.''  Thus, they will say ``Mosey house,'' but if asked whose
house that is, the answer is ``Mosey own.''  ``Co' Molsy y'own'' was
the odd reply made by Mylie to the question whose child she was
carrying.  Literally translated, this is ``Molsy's;'' \emph{Co'} is
title, \emph{y} euphonic.  An officer of a colored regiment standing
by me when the answer was made---himself born a slave---confessed that
it was mere gibberish to him.  No doubt this custom would in time
develop a regular inflectional possessive; but the establishment of
schools will soon root up all these original growths.

Very commonly, in verbs which have strong conjugations, the forms of
the past tense are used for the present; ``What make you leff we?''
``I tuk dem brudder'' (No.~30).  Past time is expressed by
\emph{been,} and less commonly \emph{done.}  ``I been kep him home two
day,'' was the explanation given for a daughter's absence from school.
``I done pit my crap in de groun'.''  Present time is made definite by
the auxiliary \emph{do} or \emph{da,} as in the refrains ``Bell da
ring,'' ``Jericho da worry me.'' (Nos.~46, 47).  ``Bubber (brother) da
hoe he tater.''  So \emph{did} occasionally: ``Nat did cuss me,''
complained one boy of another.  It is too much to say that the verbs
have no inflections, but it is true that these have nearly
disappeared.  Ask a boy where he is going, and the answer is ``gwine
crick for ketch crab'' (going into the creek to catch crabs); ask
another where the missing boy is, and the answer is the same, with
\emph{gone} instead of \emph{gwine.}  The hopeless confusion between
auxiliaries is sometimes very entertaining: as ``de-de,'' ``ain't you
know?''  ``I didn't been.''  ``De Lord is perwide'' (No.~2).  ``You'd
better pray, de worl' da [is] gwine'' (No.~14).  ``My stomach been-a
da hut me.''

Some of these sentences illustrate two other peculiarities---the
omission of auxiliaries and other small words, and the use of
\emph{for} as the sign of the infinitive.  ``Unky Taff call Co' Flora
for drop tater.''  ``Good for hold comb'' was the wisest answer found
to the teacher's question what their ears were good for.  ``Co' Benah
wan' Mr.---for tuk 'em down,'' was Gib's whispered comment when the
stubborn Venus refused to step down from a bench.  After school the
two were discovered at fisticuffs, and on being called to
account---``dat same Benah dah knock me,'' said Gib, while Venus
retorted with ``Gib cuss me in school.''

It is owing to this habit of dropping auxiliaries that the passive is
rarely if ever indicated.  You ask a man's name, and are answered,
``Ole man call John.''  ``Him mix wid him own f\"at,'' was the
description given of a paste made of bruised ground-nuts, the oil of
the nut furnishing moisture.  ``I can't certain,'' ``The door didn't
fasten,'' ``The bag won't full,'' ``Dey frighten in de dark,'' are
illustrations of every-day usage.

Proper names furnish many curious illustrations of the corruption in
pronunciation.  Many of them are impossible to explain, and it is
still only a surmise that \emph{Finnick} is derived from
\emph{Phoenix,} and \emph{Wyna} from \emph{Malvina} (the first
syllable being dropped, as in \emph{'Nelius} for \emph{Cornelius,} and
\emph{'Rullus} for \emph{Marullus.})  \emph{Hacless} is unquestionably
\emph{Hercules,} and \emph{Sack} no doubt \emph{Psyche;}
\emph{Strappan} is supposed to be \emph{Strephon.}  All these are
common names on the Sea Islands.  Names of trades, as \emph{Miller,}
\emph{Butcher,} are not uncommon.  One name that I heard of, but did
not myself meet with, was \emph{After-dark,} so called because he was
so black that ``you can't sh'um 'fo' day-clean.''

In conclusion, some actual specimens of talk, illustrating the various
points spoken of, may not be without interest.  A scene at the opening
of school:\footnote{It is proper to state that most of the materials
for this scene were furnished by Mr.~Arthur Sumner, which accounts for
the similarity of certain of the expressions to those in the dialogue
given in the September number of the Boston \emph{Freedman's Record.}}

``Charles, why did n't you come to school earlier?''  ``A-could n't
come \emph{soon} to-day, sir; de boss he sheer out clo' dis mornin'.''
``What did he give you?''  ``Me, sir? I ain't \emph{git;} de boss he
de baddest buckra ebber a-see.  De morest part ob de mens dey git
heaps o' clo'---more'n 'nuff; 'n I ain't git nuffin.''  ``Were any
other children there?''  ``Plenty chil'n, sir.  All de chil'n dah fo'
sun-up.''  ``January, you have n't brought your book.''  ``I
\emph{is,} sir; sh'um here, sir?''  ``Where is Juno?''  ``I ain't know
where he gone, sir.''  ``Where is Sam?''  ``He didn't been here.''
``Where is the little boy, John?''  ``He pick up he foot and run.''  A
new scholar is brought: ``Good mornin', maussa; I bring dis same chile
to school, sir: \emph{do} don't let 'em stay arter school done.  Here
you, gal, stan' up an' say howdy to de genlmn.  Do maussa lash 'em
well ef he don't larn he lesson.''  ``Where's your book, Tom?''
``Dunno, sir.  Some\emph{body} mus' a tief 'em.''  ``Where's your
brother?''  ``Sh'um dar?  wid bof he han' in he pocket?''  ``Billy,
have you done your sum?''  ``Yes, sir, I out 'em.''  ``Where's
Polly?''  ``Polly de-de.''  Taffy comes up.  ``Please, sir, make me
sensibble of dat word---I want to ketch 'em werry bad, sir, werry
bad.''  Hacless begins to read.  He spells, in a loud whisper, ``g-o;
g-o; g-o---can't fetch dat word, sir, nohow.''

The first day Gib appeared in school I asked him whether he could
read, and received a prompt answer in the affirmative.  So, turning to
the first page of Willson's Primer, I told him to read.  The sentence
was ``I am on,'' or something of that sort, opposite a picture of a
boy on a rocking-horse.  Gib attacked it with great volubility,
``h-r-s-e, horse.  De boy is on top ob de horse''---adding some
remarks about a chair in the background.  His eye then fell on a
picture of an eagle, and without pausing he went on, ``De raben is big
bird.''  Next he passed to a lion on the opposite page, ``D-o-g,
dog;'' but just then a cut above, representing a man and an ox, proved
too strong for him, and he proceeded to give a detailed history of the
man and the cow.  When this was completed, he took up a picture of a
boy with a paper soldiers' cap and a sword.  ``Dis man hab sword; he
tuk' e sword an' cut' e troat.''  Here I checked him, and found, as
may be expected, that he did not know a single letter.

A scene at a government auction: Henry and Titus are rivals, bidding
for a piece of ``secesh'' furniture.  Titus begins with six dollars.
``Well, Titus, I won't strain you---eight.''  ``Seven,'' says Titus.
``Ten,'' says Henry.  ``Twelve,'' says Titus.  ``And den,'' said our
informant, ``Henry bid fourteen an' tuk 'em for fifteen.''

One Day when we returned from a row on the creek, to make a call, Dick
met us with his face on a grin: ``You seen him?  you seen Miss T?
\emph{I} seen him.  I tole him you gone wid intention call on she, but
de boat didn't ready in time.  He cotch you at Mr.~H., on'y de horse
bodder him at de gate.''  One of the boys came to me one day with the
complaint, ``Dem Ma' B.~Fripp chil'n fin' one we book,'' \emph{i. e.},
those children from Mr.~T.~B.~Fripp's have found one of our books.
``'E nebber crack 'e bret,'' \emph{i.~e.,} say a word.  ``What make
you don't?''  ``Mr.~P.~didn't must.''  ``I don't know what make I
didn't answer.''  ``How do you do to-day?''  ``Stirrin,''; ``spared,''
``standin';'' ``out o' bed,'' (never ``very well.'')  Or, of a friend,
``He feel a lee better'n he been, ma'am.''

``Arter we done chaw all de hard bones and swallow all de bitter
pills,'' was part of a benediction; and the prayer at a
``praise-meeting'' asked ``dat all de white bredren an' sister what
jine praise wid we to-night might be bound up in de belly-band ob
faith.''  At a funeral in a colored regiment: ``One box o' dead meat
gone to de grave to-day---who gwine to-morrow?  Young man, who walk so
stiff---ebery step he take seem like he say, `Look out dah, groun', I
da comin'.''  The following is Strappan's view of Love.  ``Arter you
lub, you lub, you know, boss.  You can't broke lub.  Man can't broke
lub.  Lub stan'---'e ain't gwine broke.  Man hab to be berry smart for
broke lub.  Lub is a ting stan' jus' like tar; arter he stick, he
stick, he ain't gwine move.  He can't move less dan you burn him.  Hab
to kill all two arter he lub 'fo' you broke lub.''

It would be an interesting, and perhaps not very difficult inquiry, to
determine how far the peculiarities of speech of the South Carolina
negroes result from the large Huguenot element in the settlement of
that State.  It would require, however, a more exact acquaintance than
I possess with the dialects of other portions of the South, to form a
judgment of any value upon this point.  Meanwhile, I will say only
that two usages have struck me as possibly arising from this source,
the habitual lengthening of vowel sounds, and the pronunciation of
\emph{Maussa,} which may easily have been derived from
\emph{Monsieur.}  After all, traces of Huguenot influence should by
right be found among the whites, even more than the blacks.

\begin{flushright}
[W.~F.~A.]
\end{flushright}

\textsc{It} remains for the Editors to acknowledge the aid they have
received in making this compilation.  To
Col.~\textsc{T.~W.~Higginson}, above all others, they are indebted for
friendly encouragement and for direct and indirect contributions to
their original stock of songs.  From first to last he has manifested
the kindest interest in their undertaking, constantly suggesting the
names of persons likely to afford them information, and improving
every opportunity to procure them material.  As soon as his own
valuable collection had appeared in the \emph{Atlantic Monthly,} he
freely made it over to them with a liberality which was promptly
confirmed by his publishers, Messrs. \textsc{Ticknor \& Fields.}  It
is but little to say that without his co-operation this \emph{Lyra
  Africana} would have lacked greatly of its present completeness and
worth.  Through him we have profited by the cheerful assistance of
Mrs.~\textsc{Charles J.~Bowen,} Lieut.-Colonel
\textsc{C.~T.~Trowbridge,} Capt.~\textsc{James S.~Rogers,}
Rev.~\textsc{Horace James,} Capt. \textsc{Geo.~S.~Barton,} Miss
\textsc{Lucy Gibbons,} Mr.~\textsc{William A.~Baker,}
Mr.~\textsc{T.~E.~Ruggles,} and Mr.~\textsc{James Schouler.}  Our
thanks are also due for contributions, of which we have availed
ourselves, to Dr.~\textsc{William A.~Hammond,}
Mr.~\textsc{Geo.~H.~Allan,} Lt.-Col.~\textsc{Wm.~Lee Apthorp,}
Mr.~\textsc{Kane O'Donnel,} Mr.~\textsc{E.~J.~Snow,} Miss
\textsc{Charlotte L.~Forten,} Miss \textsc{Laura M.~Towne,} and Miss
\textsc{Ellen Murray;} and for criticisms, suggestions,
communications, and unused but not unappreciated contributions, to
Mr.~\textsc{John R.~Dennett,} Miss \textsc{Annie Mitchell,}
Mr.~\textsc{Reuben Tomlinson,} Mr.~\textsc{Arthur Sumner,}
Mr.~\textsc{N.~C.~Dennett,} Miss \textsc{Mary Ellen Peirce,}
Maj-Gen.~\textsc{Wager Swayne,} Miss \textsc{Maria W.~Benton,}
Prof.~\textsc{J.~Silsby,} Rev.~\textsc{John L.~McKim,}
Mr.~\textsc{Albert Griffin,} Mr.~\textsc{A.~S.~Jenks,}
Mr.~\textsc{E.~H.~Hawkes,} Rev.~\textsc{H.~C.~Trumbull,}
Rev.~\textsc{J.~K.~Hosmer,} Rev.~\textsc{F.~N.~Knapp,}
Brev.~Maj.-Gen.~\textsc{Truman Seymour,} Maj.-Gen.~\textsc{James
  H.~Wilson,} Mr.~\textsc{J.~H.~Palmer,} and others; and, finally, to
the editors of various newspapers who gratuitously announced the
forthcoming volume.

Conscious of many imperfections in this, the result of not
inconsiderable joint labor for nearly a year, the Editors submit it,
nevertheless, to the public judgment, in the belief that it will be
pronounced deserving of even greater pains and of permanent
preservation.

\begin{flushright}
\textsc{William Francis Allen,}\\
\textsc{Charles Pickard Ware,}\\
\textsc{Lucy McKim Garrison.}
\end{flushright}


\chapter{Directions for Singing.}

\textsc{In} addition to those already given in the Introduction, the
following explanations may be of assistance:

Where all the words are printed with the music, there will probably be
little difficulty in reading the songs; but where there are other
words printed below the music, it will often be a question to which
part of the tune these words belong, and how the refrain and the
chorus are to be brought in.

It will be noticed that the words of most of the songs arrange
themselves into stanzas of four lines each.  Of these some are
\emph{refrain}, and some are \emph{verse} proper.  The most common
arrangement gives the second and fourth lines to the refrain, and the
first and third to the verse; and in this case the third line may be a
repetition of the first, or may have different words.  Often, however,
the refrain occupies only one line, the verse occupying the other
three; while in one or two songs the verse is only one line, while the
refrain is three lines in length.  The refrain is repeated with each
stanza: the words of the verse are changed at the pleasure of the
leader, or fugleman, who sings either well-known words, or, if he is
gifted that way, invents verses as the song goes on.

In addition to the stanza, some of the songs have a chorus, which
usually consists of a fixed set of words, though in some of the songs
the chorus is a good deal varied.  The refrain of the main stanza
often appears in the chorus.  The stanza can always be distinguished
from the chorus, in those songs which have more than one stanza, by
the figure ``1'' placed before the stanza which is printed with the
music; the verses below being numbered on ``2,'' ``3,'' ``4,'' \&c.
In a few cases the first verse below the music is numbered ``3;'' this
occurs when two verses have been printed above in the music, instead
of the first verse being repeated.  When the chorus has a variety of
words, the additional verses are printed below without numbers.

In the following list the first fifty tunes in the collection are
classified according to the peculiarity of their division into verse
and refrain.  It is hoped that this will help to remove all
obscurities with which the reader may be embarrassed.

%% FIXME: Make table, or lists or whatever.

No explanation is needed for Nos.~2, 12, 13, 18, 22--26, 34, 36,
38--43.

Single line and refrain, 27, 35.

Single line and refrain with chorus, 6, 29.

Stanza of 4 lines:

No refrain; chorus, 11.

4th line refrain; introduction, 7.

4th line refrain; chorus, 8, 9, 10, 15, 37, 45.

1st and 2d lines verse, 3d and 4th refrain; chorus, 1, 4.

1st and 3d lines verse, 2d and 4th refrain, 14, 17, 20, 28, 31, 32,
33, 47, 48, 49, 50.

1st and 3d lines verse, 2d and 4th refrain; double, 21.

1st and 3d lines verse, 2d and 4th refrain; chorus, 3, 30, 44.

1st and 3d lines verse, 2d and 4th refrain; introduction, 46.

1st line verse; chorus, 5.

1st line verse (double); chorus, 19.

3d line verse, 16.

%% FIXME: Small quarter notes!

As regards the \emph{tempo}, most of the tunes are in 2--4 time, and
in most of these FIXME=100---(say) 100--120.  The spirit of the music
will determine the \emph{tempo} within these limits.  The slower tunes
are 1, 3, 9, 17, 21, etc.  No.~2 is about FIXME=160--180, and perhaps
had better have been written in 3--8.  So No.~13 would be better in
2--4; as it is, the FIXME=160--170.  No.~24 should be read as if
divided in 2--4, with FIXME=100.  The \emph{tempo} of the rowing tunes
has been already indicated.

The pitch has generally been accommodated to voices of medium range.



\cleardoublepage
\pagenumbering{arabic}

\part{South-Eastern Slave States: including South Carolina, Georgia and the Sea Islands.}

%%
%% Setup for the score part of the book.
%%


%% We don't want no indentation.
\newlength{\parindentsave}
\setlength{\parindentsave}{\parindent}
\setlength{\parindent}{0em}


%% \changepage{textheight}{textwidth}{evensidemargin}{oddsidemargin}{columnsep}{topmargin}{headheight}{headsep}{footskip}
\changepage{180pt}{80pt}{-65pt}{-15pt}{}{-90pt}{}{}{}

%% Font used for stanzas (magically extracted from lilypond-book output).
% New Century Schoolbook, 11pt.
\newcommand{\stanzafont}{\normalsize}

%\font\smallerfont=cmb10
%\font\smallerfont=cmb7

%% We reduce the vertical space between staff lines as much as plausible.
\def\interscoreline{\vskip 1pt}


%% Some environments.
\newenvironment{song}
{
  % The -13pt is the first X coordinate of the bounding box in the EPS
  % files generated by lilypond.  The effect is to align the staff with
  % the left margin, ignoring the measure numbering.
  \begin{list}{}{\leftmargin-13pt}\item{}
}
{
  \end{list}
  % Because lilypond wraps the score sheet in a block ``{}'', the
  % stanzas following are separated by some extra white space.  Rather
  % than trying to battle this (how?), we just skip back up and go on.
  % The -20pt is somewhat arbitrary and unrelated to the above -13pt.
  % Note that also for whitespace reasons we have to add footnotetext
  % instructions inside the song environment.
  \vspace{-20pt}
}

\newenvironment{songpart}
{
  % The -13pt is the first X coordinate of the bounding box in the EPS
  % files generated by lilypond.  The effect is to align the staff with
  % the left margin, ignoring the measure numbering.
  \begin{list}{}{\leftmargin-13pt}\item{}
}
{
  \end{list}
  % Because lilypond wraps the score sheet in a block ``{}'', the
  % stanzas following are separated by some extra white space.  Rather
  % than trying to battle this (how?), we just skip back up and go on.
  % The -20pt is somewhat arbitrary and unrelated to the above -13pt.
  % Note that also for whitespace reasons we have to add footnotetext
  % instructions inside the song environment.
  \vspace{-31pt}
}

\newenvironment{stanza}
{
  \begin{enumerate}
  \stanzafont
}
{
  \end{enumerate}
}

\newlength{\parindenttmp}
\newenvironment{extra}
{
  \setlength{\parindenttmp}{\parindent}
  \setlength{\parindent}{\parindentsave}
  \small
}
{
  \setlength{\parindent}{\parindenttmp}
}





\section{Roll, Jordan, roll.}
\thispagestyle{empty}

% Normal is 276pt.
\begin{song}
  \lilypondfile[staffsize=18,line-width=356\pt]{001.ly}

  \footnotetext[1]{Parson Fuller, Deacon Henshaw, Brudder Mosey, Massa
    Linkum, \&c.}
\end{song}


\begin{stanza}
\item[2.]
  Little chil'en, learn to fear de Lord, \\
  And let your days be long; \\
  Roll, Jordan, \&c.

\item[3.]
  O, let no false nor spiteful word \\
  Be found upon your tongue; \\
  Roll, Jordan, \&c.
\end{stanza}

\begin{extra}
  [This spiritual probably extends from South Carolina to Florida, and
    is one of the best known and noblest of the songs.]
  
  [The following variation appears above bars 4--6 in the original
    print. - Ed.]

  \medskip
  \begin{song}
    \begin[staffsize=14,notime,ragged-right]{lilypond}
    <<
      \context Voice
      {
        \override Staff.VerticalAxisGroup #'minimum-Y-extent = #'(-1 . 1)

        \autoBeamOff
	       
        \clef violin
        \key d \major
        \time 2/4

        \set Score.currentBarNumber = #4
        %%\override Score.BarNumber  #'padding = #3
	       
        a'4( f') \bar "|" g'4 g'8 a' \bar "|" f'4 d'8 f' \bar "|"
      }
      \lyricsto "" \new Lyrics
      {
        \override LyricText #'font-size = #0
        \override StanzaNumber #'font-size = #-1

        roll; Roll, Jor -- dan, Roll, Jor -- dan,
      }
    >>
  \end{lilypond}
  \end{song}
\end{extra}


\newpage
\section{Jehovah, Hallelujah.}
\thispagestyle{empty}

\begin{song}
  \lilypondfile[staffsize=18,line-width=356\pt]{002.ly}

  \footnotetext[1]{Will provide.}
  \footnotetext[2]{Hanno.}
\end{song}



\newpage
\section{I hear from Heaven to-day.}
\thispagestyle{empty}

%% FIXME: Fix variations.
\begin{song}
  \lilypondfile[staffsize=18,line-width=356\pt]{003.ly}

  \footnotetext[1]{Travel.}
  \footnotetext[2]{My brudder, Brudder Jacob, Sister Mary.}
\end{song}

\begin{stanza}
\item[2.]
  A baby born in Bethlehem, \\
  And I yearde, \&c.

\item[3.]
  De trumpet sound in de oder bright land.\footnote[3]{World.}
  
\item[4.]
  My name is called and I must go.

\item[5.]
  De bell is a-ringin' in de oder bright world.
\end{stanza}


\newpage
\section{Blow your trumpet, Gabriel.}
\thispagestyle{empty}

\begin{song}
  \lilypondfile[staffsize=18,line-width=356\pt]{004.ly}
\end{song}

\begin{stanza}
\item[2.]
  Paul and Silas, bound in jail, \\
  Sing God's praise both night and day; \\
  And I hope, \&c.
\end{stanza}

\begin{extra}
  [This hymn is sung in Virginia in nearly the same form. The
    following minor variation is given by Mrs.~Bowen, as heard by her
    in Charleston, some twenty-five years ago:]

  \begin{song}
    \begin[staffsize=14,line-width=316\pt]{lilypond}
    \score
    {
      \new Staff {
        <<

          \context Voice
          {
	    \set Staff.midiInstrument = "acoustic grand"
            \override Staff.VerticalAxisGroup #'minimum-Y-extent = #'(0 . 0)
	
	    \autoBeamOff
	    
	    \time 2/4
	    \clef violin
	    %% EDITED: Key changed to G major from G minor.
	    \key g \major

	    {
	      \override Stem #'neutral-direction = #1
	      bes'8 bes' a' a' | bes'8 bes' a' r | bes'8 bes' a' g' |
	      fis'8 g' a' a'16 a' |
	      \break
	      bes'8 bes' a' a' | g'8. e'16 d'8 e'16 fis' |
	      g'8. bes'16 a'8 a' | g'4 r8 a' |
	      \break
	      b'8. b'16 g'8
	      \override Stem #'neutral-direction = #-1
	      b' |
	      \override Stem #'neutral-direction = #1
	      c''8 d''4 r8 |
	      e''8. d''16 c''8 e'' | e''8 d'' r a'16 a' |
	      \break
	      bes'8 bes' a' a' |
	      %% EDITED: I changed the original g'4. to g'8. to make
	      %% it fit into the measure.
	      g'8. e'16 d'8 e'16 fis' | g'8. bes'16 a'8 a' | g'4 r
	    }
	  }
	  \lyricsto "" \new Lyrics
	  {
            \override LyricText #'font-size = #0
            \override StanzaNumber #'font-size = #-1

	    Paul and Si -- las, bound in jail,
	    Chris -- tians pray both night and day,
	    And I hope dat trump might blow me home
	    To my new Je -- ru -- sa -- lem.
	    So blow de trum -- pet, Gab -- riel,
	    Blow de trum -- pet loud -- er,
	    And I hope dat trump might blow me home
	    To my new Je -- ru -- sa -- lem.
	  }
	  >>
      }
      
      \layout { indent = 0.0 }
    }
  \end{lilypond}
  \end{song}
\end{extra}

%% EDITED \bigskip
%% [In the original book, this variation was set in the key of G minor.
%% However, with hindsight---just notice the major third in the second
%% part of the piece---it can be better interpreted as being in the key
%% of G major with the use of blue notes. - Ed.]


\newpage
\section{Praise, member.}
\thispagestyle{empty}

\begin{song}
  \lilypondfile[staffsize=18,line-width=356\pt]{005.ly}

  \footnotetext[1]{Believer.}
  \footnotetext[2]{Religion so sweet.}
  \footnotetext[3]{Shore.}
  \footnotetext[4]{Stream, Fight.}
\end{song}

\begin{stanza}
\item[2.]
  O soldier's fight is a good old fight, \\
  And I hain't, \&c.

\item[3.]
  O I look to de East, and I look to de West.

\item[4.]
  O I wheel to de right, and I wheel to de left.
\end{stanza}

\begin{extra}
  [The last verse is varied in several different ways; Col.~Higginson
    gives, ``There's a hill on my leff, an' he catch on my right,''
    and says, ``I could get no explanation of this last riddle,
    except, `Dat mean, if you go on de leff, you go to 'struction, and
    if you go on de right, go to God, for sure.'{}'' Miss Forten
    gives, ``I hop on my right an' I catch on my leff,'' and supposes
    ``that some peculiar motion of the body formed the original
    accompaniment of the song, but has now fallen into disuse.''
    Lt.~Col.~Trowbridge heard this hymn sung among the colored people
    of Brooklyn, N.~Y., several years ago.]
\end{extra}


\newpage
\section{Wrestle on, Jacob.}
\thispagestyle{empty}

\begin{song}
  \lilypondfile[staffsize=18,line-width=356\pt]{006.ly}

  \footnotetext[1]{My sister, Brudder Jacky, All de member.}
  \footnotetext[2]{I would not let him go.}
  \footnotetext[3]{Lord I.}
\end{song}

\begin{stanza}
\item[2.] I will not let you go, my Lord.

\item[3.] Fisherman Peter out at sea.

\item[4.] He cast\footnote[4]{Fish.} all night and he
cast\footnotemark[4] all day.

\item[5.] He\footnote[5]{I.} catch no fish, but he\footnotemark[5] catch
some soul.

\item[6.] Jacob hang from a tremblin' limb.
\end{stanza}

\begin{extra}
  [This is also sung in Maryland and Virginia, in a slightly modified
    form.  A Virginia verse is,---
    \begin{quote}
      I looked to the East at the breaking of the day, \\
%% FIXME: The University of North Carolina at Chapel Hill scan has
%% ``when'' not ``went''.  What's correct?
      The old ship of Zion went sailing away.]
    \end{quote}
\end{extra}


\newpage
\section{The Lonesome Valley.}
\thispagestyle{empty}

\begin{song}
  \lilypondfile[staffsize=18,line-width=356\pt]{007.ly}

  \footnotetext[1]{Sister Katy, etc.}
\end{song}

\begin{stanza}
\item[2.]
  O feed on milk and honey.

\item[3.]
  O John he write de letter.

\item[4.]
  And Mary and Marta read 'em.
\end{stanza}

\begin{extra}
  [``{}`De valley,' and `de lonesome valley' were familiar words in
    their religious experience.  To descend into that region implied
    the same process with the `anxious-seat' of the camp-meeting.
    When a young girl was supposed to enter it, she bound a
    handkerchief by a peculiar knot over her head, and made it a point
    of honor not to change a single garment till the day of her
    baptism, so that she was sure of being in physical readiness for
    the cleansing rite, whatever her spiritual mood might be.  More
    than once, in noticing a damsel thus mystically kerchiefed, I have
    asked some dusky attendant its
    %% EDITED: The dash was made longer.
    meaning, and have received the unfailing answer,---framed with
    their usual indifference to the genders of pronouns,---`He in de
    lonesome valley, sa.'{}''---\emph{Col.~Higginson.}]
\end{extra}


\newpage
\section{I can't stay behind.}
\thispagestyle{empty}

\begin{song}
  \lilypondfile[staffsize=18,line-width=356\pt]{008.ly}

  \footnotetext[1]{For you.}
\end{song}

\begin{stanza}
\item[2.]
  I been all around, I been all around, \\
  Been all around de Heaven, my Lord.

\item[3.]
  I've searched every room---in de Heaven, my Lord.\footnote[2]{And Heaven
  all around.}

\item[4.]
  De angels singin'\footnote[3]{Crowned}---all around de trone.

\item[5.]
  My Fader call---and I must go.

\item[6.]
  Sto-back,\footnote[4]{``Sto-back'' means ``Shout backwards''} member;
  sto-back, member.
\end{stanza}


\newpage
\thispagestyle{empty}

\begin{extra}
  [This ``shout'' is very widely spread, and variously sung.  In
    Charleston it is simpler in its movement, and the refrain is ``I
    can't stay away.''  In Edgefield it is expostulating: ``Don't stay
    away, my mudder.''  Col.~Higginson gives the following version, as
    sung in his regiment:

    %% FIXME: This is not small as the rest in the extra environment.
    \begin{stanza}
    \item[]
      ``O, my mudder is gone!  my mudder is gone! \\
      My mudder is gone into heaven, my Lord! \\
      I can't stay behind! \\
      Dere's room in dar, room in dar. \\
      Room in dar, in de heaven, my Lord! \\
      I can't stay behind. \\
      Can't stay behind, my dear, \\
      I can't stay behind!
      
    \item[]
      ``O, my fader is gone! \&c.

    \item[]
      ``O, de angels are gone! \&c.
      
    \item[]
      ``O, I'se been on de road!  I'se been on de road! \\
      I'se been on de road into heaven, my Lord! \\
      I can't stay behind! \\
      O, room in dar, room in dar, \\
      Room in dar, in de heaven, my Lord! \\
      I can't stay behind!''
    \end{stanza}
    
    Lt.~Col.~Trowbridge is of opinion that it was brought from
    Florida, as he first heard it in Dec, 1862, from a boat-load of
    Florida soldiers brought up by Lt.~Col.~Billings.  It was not
    heard by Mr.~Ware at Coffin's Point until that winter.  It seems
    hardly likely, however, that it could have made its way to
    Charleston and Edgefield since that time.  The air became
    ``immensely popular'' in the regiment, and was soon adopted for
    military purposes, so that the class leaders indignantly
    complained of ``the drum corps using de Lord's chune.'']
\end{extra}


\newpage
\section{Poor Rosy.}
\thispagestyle{empty}

\begin{song}
  \lilypondfile[staffsize=18,line-width=356\pt]{009.ly}

  \footnotetext[1]{Poor C\ae{}sar, poor boy.}
\end{song}

\begin{stanza}
\item[2.]
  Got hard trial in my way, \emph{(ter)} \\
  Heav'n shall-a be my home.

%% FIXME: There is an indentation here that must be removed.
%% And every second line should be indented.
  $\left.
  \begin{array}{l}
    \mbox{O when I talk\footnote[2]{Walk.}, I talk\footnotemark[2] wid God,} \\
    \mbox{Heav'n shall-a be my home.}
  \end{array}
  \right\}
  \mbox{\emph{(bis)}}$

\item[3.]
  I dunno what de people\footnote[3]{Massa.} want of me, \emph{(ter)} \\
  Heav'n shall-a be my home.
\end{stanza}

\begin{extra}
  [This song ranks with ``Roll, Jordan,'' in dignity and favor.  The
    following variation of the second part was heard at ``The Oaks:'']

  %% EDITED: I changed the key of the variation from C major to E minor.
  %% Apparently the sharp sign was missing.
  \medskip
  \begin{song}
    \begin[staffsize=14]{lilypond}
    <<
      \context Voice
      {
        \override Staff.VerticalAxisGroup #'minimum-Y-extent = #'(-1 . 1)
	       
	\autoBeamOff
	   
	\clef violin
	\key e \minor
	\time 2/4

	%% FIXME: Make the bar number pop up at the beginning of the line.
	\set Score.currentBarNumber = #9
	%%\override Score.BarNumber #'padding = #3

	\repeat volta 2
	{
	  \partial 8 e'8 \bar "|"
	  e'8 e' g' e' \bar "|"
	  d'8 d' b4 \bar "|"
	  g'16 g' g' g' g'8 fis' \bar "|" e'4 r8
	}
      }
      \lyricsto "" \new Lyrics
      {
        \override LyricText #'font-size = #0
        \override StanzaNumber #'font-size = #-1

	\repeat fold 2 {}
	\alternative
	{
	  {
	    Be -- fore I stay in hell one day,
	    Heav -- en shall -- a be my home;
	  }
	  {
	    I sing and pray my soul a -- way,
	    Heav -- en shall -- a be my home.
	  }
	}
      }
    >>
  \end{lilypond}
  \end{song}
\end{extra}


\newpage
\section{The Trouble of the World.}
\thispagestyle{empty}

\begin{song}
  \lilypondfile[staffsize=18,line-width=356\pt]{010.ly}

  \footnotetext[1]{O you ought to be.}
  \footnotetext[2]{My sister, My mudder, etc.}
\end{song}

\begin{stanza}
\item[2.]
I ask de Lord how long I hold 'em, \emph{(ter)} \\
Hold 'em to de end.

\item[3.]
My sins so heavy I can't get along, Ah! \&c.

\item[4.]
I cast my sins in de middle of de sea, Ah! \&c.
\end{stanza}

\begin{extra}
  [This is perhaps as good a rendering of this strange song as can be
    given.  The difficulty is in the time, which is rapid, hurried and
    irregular to a degree which is very hard to imitate and impossible
    to represent in notes.  The following is sung in Savannah, with
    the same refrain, ``Trouble of the world:'']

  %% FIXME: Song very close to footnotes.
  \begin{song}
    \begin[staffsize=14]{lilypond}
    <<
      \context Voice
      {
        \override Staff.VerticalAxisGroup #'minimum-Y-extent = #'(-1 . 1)

	\autoBeamOff
	   
	\clef violin
	\key g \major
	\time 2/4

	\override Stem #'neutral-direction = #1
	\repeat volta 2
	{
	  \partial 8 d'8 | b'8 b'a ' a' | g'8 g' fis'4 | e'2 |
	  fis'8 fis' d' d' |
	  \break
	  fis'8 fis' fis' fis' | g'8 g' c''4 |
	  b'4 a'8 a' | g'4 r8
	}
      }
      \lyricsto "" \new Lyrics
      {
        \override LyricText #'font-size = #0
        \override StanzaNumber #'font-size = #-1

	I wish I was in ju -- bi -- lee, Ha, ju -- bi -- lee;
	I wish I was in ju -- bi -- lee, Roll, Jor -- dan, roll.
      }
    >>
  \end{lilypond}
  \end{song}
\end{extra}


\newpage
\section{There's a meeting here to-night.}
\thispagestyle{empty}

\begin{song}
  \lilypondfile[staffsize=18,line-width=356\pt]{011.ly}
\end{song}

\begin{stanza}
\item[2.]
  Brudder John was a writer, he write de laws of God; \\
  Sister Mary say to brudder John, ``Brudder John, don't write no more.'' \\
  Dere's a meeting here to-night, Oh!  (Brudder Sandy,) \emph{(bis)} \\
  Dere's a meeting here to-night, \\
  I hope to meet again.
\end{stanza}

\begin{extra}
  [Mrs.~Bowen gives us the following beautiful variation, as sung in
    Charleston:]

  \medskip
  \begin{song}
    \begin[staffsize=14,line-width=316\pt]{lilypond}
    \score
    {
      \new Staff {
      <<
        \context Voice
	{
	  \set Staff.midiInstrument = "acoustic grand"
          \override Staff.VerticalAxisGroup #'minimum-Y-extent = #'(0 . 0)
	
	  \autoBeamOff
	  
	  \time 2/4
	  \clef violin
	  \key d \major

	  {
	    \partial 4 a'4 |
	    d''8 d''16 d''16 d''8 e'' | d''8 cis'' r a'16 a' |
	    b'8 b' d''8. b'16 | b'8 a' r a'16 a' |
	    \break
	    \override Stem #'neutral-direction = #1
	    d''8 a' g' b' | a'8 fis' r fis'16 fis'16 |
	    a'8 g' e'8. fis'16 | d'4 r8 d'16 e' |
	    \break
	    fis'8 fis' fis'8. e'16 | a'4 r8 g'16 fis' |
	    e'8 e' e'8. d'16 | e'4 e''8-^ cis''8 |
	    \break
	    d''8 a' g' b' | a'8 fis' r fis'16 fis' |
	    a'8 g' e'8. fis'16 | d'4 \bar "||"
	  }
	}
	\lyricsto "" \new Lyrics
	{
          \override LyricText #'font-size = #0
          \override StanzaNumber #'font-size = #-1

	  I see brud -- der Mo -- ses yon -- der,
	  And I think I ought to know him,
	  For I know him by his gar -- ment,
	  He's a bless -- ing here to -- night;
	  He's a bless -- ing here to -- night,
	  He's a bless -- ing here to -- night,
	  And I think I ought to know him,
	  He's a bless -- ing here to -- night.
	}
      >>
    }
    \layout { indent = 0.0 }
  }
  \end{lilypond}
  \end{song}
\end{extra}


\newpage
\section{Hold your light.}
\thispagestyle{empty}

\begin{song}
  \lilypondfile[staffsize=18,line-width=356\pt]{012.ly}

  \footnotetext[1]{Long o'.}
  \footnotetext[2]{All de member, Turn seeker.}
\end{song}

\newpage
\section{Happy Morning.}
\thispagestyle{empty}

\begin{song}
  \lilypondfile[staffsize=18,line-width=356\pt]{013.ly}

  \footnotetext[1]{Doubt no more, Thomas.}
  \footnotetext[2]{Glorious, Sunday.}
  \footnotetext[3]{O what a happy Sunday.}
\end{song}

\newpage
\section{No man can hinder me.}
\thispagestyle{empty}

\begin{song}
  \lilypondfile[staffsize=18,line-width=356\pt]{014.ly}
\end{song}

\begin{stanza}
\item[3.]
  Jesus make de dumb to speak.

\item[4.]
  Jesus make de cripple walk.

\item[5.]
  Jesus give de blind his sight.

\item[6.]
  Jesus do most anyting.

\item[7.]
  Rise, poor Lajarush, from de tomb.

\item[8.]
  Satan ride an iron-gray horse.

\item[9.]
  King Jesus ride a milk-white horse.
\end{stanza}

\begin{extra}
  \emph{Variation.}

  \begin{song}
    \begin[staffsize=14,line-width=316\pt]{lilypond}
    \score
    {
      \new Staff {
	<<
	\context Voice
	{
	  \set Staff.midiInstrument = "acoustic grand"
          \override Staff.VerticalAxisGroup #'minimum-Y-extent = #'(0 . 0)
	
	  \autoBeamOff
	  
	  \time 2/4
	  \clef violin
	  \key c \major
	  
          {
	    \partial 8 g'8 |
	    g'8 g'8 g'8. g'16 | a'8 d'' c'' r |
	    %% FIXME: Merge note heads.
	    << { \stemUp a'4 a'8. g'16 \stemNeutral }
	    << \\ { \autoBeamOff \stemDown a'8 a'4 g'8 } >> >> |
	    a'8 b' c'' d'' | e''8 e'' c'' g' | c''8 g' e' r |
	    << { \stemUp g'4 e'8. c'16 \stemNeutral }
	    << \\ { \autoBeamOff \stemDown g'8 e'4 c'8 } >> >> |
	    d'8 c' c' r | \bar "||"
	  }
	}
        \lyricsto "" \new Lyrics
	{
          \override LyricText #'font-size = #0
          \override StanzaNumber #'font-size = #-1

	  You'd bet -- ter pray, de world da gwine, No man can hin -- der me!
	  De Lord have mer -- cy on my soul, No man can hin -- der me!
      }
    >>
  }
  \layout { indent = 0.0 }
}
  \end{lilypond}
  \end{song}
\end{extra}


\newpage
\section{Lord, remember me.}
\thispagestyle{empty}

\begin{song}
  \lilypondfile[staffsize=18,line-width=356\pt]{015.ly}

  \footnotetext[1]{Do.}
  \footnotetext[2]{I pray (cry) to de Lord.}
\end{song}

\begin{stanza}
\item[2.]
  I want to die like-a Jesus die, \\
  And he die wid a free good will, \\
  I lay out in de grave and I stretchee out e arms, \\
  Do, Lord, remember me.
\end{stanza}


\newpage
\section{Not weary yet.}
\thispagestyle{empty}

\begin{song}
  \lilypondfile[staffsize=18,line-width=356\pt]{016.ly}

  \footnotetext[1]{Sister Mary.}
\end{song}

\begin{stanza}
\item[2.]
  Since I been in de field to\footnote[2]{Been-a.} fight.

\item[3.]
  I have a heaven to maintain.

\item[4.]
  De bond of faith are on my soul.

\item[5.]
  Ole Satan toss a ball at me.

\item[6.]
  Him tink de ball would hit my soul.

\item[7.]
  De ball for hell and I for heaven.
\end{stanza}


\newpage
\section{Religion so sweet.}
\thispagestyle{empty}

\begin{song}
  \lilypondfile[staffsize=18,line-width=356\pt]{017.ly}
\end{song}

\begin{stanza}
\item[3.]
  Religion make you happy.\footnote[1]{Humble.}

\item[4.]
  Religion gib me patience.\footnote[2]{Honor, Comfort.}

\item[5.]
  O member, get religion.

\item[6.]
  I long time been a-huntin'.

\item[7.]
  I seekin' for my fortune.

\item[8.]
  O I gwine to meet my Savior.

\item[9.]
  Gwine to tell him 'bout my trials.

\item[10.]
  Dey call me boastin' member.

\item[11.]
  Dey call me turnback\footnote[3]{Lyin', 'ceitful.} Christian.

\item[12.]
  Dey call me 'struction maker.

\item[13.]
  But I don't care what dey call me.

\item[14.]
  Lord, trial 'longs to a Christian.

\item[15.]
  O tell me 'bout religion.

\item[16.]
  I weep for Mary and Marta.

\item[17.]
%% EDITED: Added ``him.''
  I seek my Lord and I find him.
\end{stanza}


\newpage
\section{Hunting for the Lord.}
\thispagestyle{empty}

\begin{song}
  \lilypondfile[staffsize=18,line-width=356\pt]{018.ly}
\end{song}


\newpage
\section{Go in the wilderness.}
\thispagestyle{empty}

\begin{song}
  \lilypondfile[staffsize=18,line-width=356\pt]{019.ly}

  \footnotetext[1]{To.}
\end{song}

\begin{multicols}{2}
\begin{stanza}
%% FIXME: Where is stanzas 2?
\item[3.]
  You want to be a Christian.

\item[4.]
  You want to get religion.

\item[5.]
  If you spec' to be converted.

\item[6.]
  O weepin' Mary.

\item[7.]
  'Flicted sister.

\item[8.]
  Say, ain't you a member?

\item[9.]
  Half-done Christian.

\item[10.]
  Come, backslider.

\item[11.]
  Baptist member.

\item[12.]
  O seek, brudder Bristol.

\item[13.]
  Jesus a waitin' to meet you in de wilderness.
\end{stanza}
\end{multicols}

\begin{extra}
  [The second part of this spiritual is the familiar Methodist hymn
    ``Ain't I glad I got out of the wilderness!''\ and may be the
    original.  The first part is very beautiful, and appears to be
    peculiar to the Sea Islands.]
\end{extra}


\newpage
\section{Tell my Jesus ``Morning.''}
\thispagestyle{empty}

\begin{song}
  \lilypondfile[staffsize=18,line-width=356\pt]{020.ly}\

  \footnotetext[1]{Morning.}
\end{song}

\begin{stanza}
\item[2.]
  Mornin', Hester, mornin' gal, \\
  Tell my Jesus, \&c.
\end{stanza}

\begin{extra}
\emph{Variation to first line.}

\begin{song}
  \begin[staffsize=14]{lilypond}
  <<
  \context Voice
  {
    \override Staff.VerticalAxisGroup #'minimum-Y-extent = #'(-1 . 1)
    
    \autoBeamOff
    
    \clef violin
    \key g \major
    \time 2/4
    
    b'8 a'16 g' b'8 g' | b'8 g'16 g' e'8 g' |
  }
  \lyricsto "" \new Lyrics
  {
    \override LyricText #'font-size = #0
    \override StanzaNumber #'font-size = #-1

    Pray To -- ny, pray boy, you got de or -- der;
  }
  >>
\end{lilypond}
\end{song}

\smallskip
\end{extra}
\begin{stanza}
\small
\item[2.]
  Say, brudder Sammy, you got de order, \\
  Tell my Jesus, \&c.

\item[3.]
  You got de order, and I got de order.
\end{stanza}


\newpage
\section{The Graveyard.}
\thispagestyle{empty}

\begin{song}
  \lilypondfile[staffsize=18,line-width=356\pt]{021.ly}

  \footnotetext[1]{Sing glory, Graveyard.}
\end{song}

\begin{stanza}
\item[3.]
  O graveyard, ought to know me.

\item[4.]
  O grass grow in de graveyard.

\item[5.]
  O I reel\footnote[2]{Shout, Wheel.} and I rock in de graveyard.

\item[6.]
  O I walk and I toss wid Jesus.

\item[7.]
  My mudder reel and-a toss wid de fever.

\item[8.]
  I have a grandmudder in de graveyard.

\item[9.]
  O where d'ye tink I find 'em?\footnote[3]{\emph{i.e.,} religion; see Preface.}

\item[10.]
  I find 'em, Lord, in de graveyard.

\item[11.]
  (Member,) I wheel, and I rock, and I gwine home.

\item[12.]
  (Brudder Sammy) O 'peat dat story over.
\end{stanza}

\begin{extra}
\emph{Variation to Verse 3.}

\medskip
\begin{song}
  \begin[staffsize=14,notime,ragged-right]{lilypond}
    <<
    \context Voice
	 {
           \override Staff.VerticalAxisGroup #'minimum-Y-extent = #'(-1 . 1)

	   \autoBeamOff
	   
	   \clef violin
	   \key f \major
	   \time 2/4

		 {
		   \partial 8 c''8 | c''8. c''16 c''8 c'' | \partial 4 d''8 c''
		 }
	 }
	 \lyricsto "" \new Lyrics
	 {
           \override LyricText #'font-size = #0
           \override StanzaNumber #'font-size = #-1

	   Grave -- yard, you ought to know me.
	 }
  >>
\end{lilypond}
\end{song}
\end{extra}


\newpage
\section{John, John, of the Holy Order.}
\thispagestyle{empty}

\begin{song}
  \lilypondfile[staffsize=18,line-width=356\pt]{022.ly}

  \footnotetext[1]{John, John, de holy Baptist.}
  \footnotetext[2]{Paul and Silas, bound in jail.}
\end{song}

\newpage
\thispagestyle{empty}

\begin{extra}
  [These words were sung at Hilton Head to the second and third parts:

    \begin{stanza}
    \item[]
      I went down sing polka, and I ax him for my Saviour; \\
      I wonder de angel told me Jesus gone along before. \\
      I mourn, I pray, although you move so slow; \\
      I wonder, \&c.
    \end{stanza}

    The regularity and elaborateness of this hymn lead one at first to
    suspect its genuineness.  The question seems, however, to be settled
    by two very interesting and undoubted variations from North Carolina
    and Georgia.  The following words were sung at Augusta, but we have
    not been able to obtain the tune, which is entirely unlike that given
    above.  For the North Carolina variation, see No.~100.  Both, as will
    be seen, omit the second part, and a comparison of the two shows that
    %% EDITED: Inserted a period after ``altar''.
    the enigmatical word ``order'' should undoubtedly be ``altar''.  The
    North Carolina tune has the first part quite different from the Port
    Royal tune, the last very similar to it.

    \begin{stanza}
    \item[]
      Oh John, John, de holy member, \\
      Sittin' on de golden ban'. \\
      O worldy, worldy, let him be, \\
      Let him be, let him be; \\
      Worldly, worldly, let him be, \\
      Sittin' on de golden ban'.]
    \end{stanza}
\end{extra}

\newpage
\section{I saw the beam in my sister's eye.}
\thispagestyle{empty}

\begin{song}
  \lilypondfile[staffsize=18,line-width=356\pt]{023.ly}

  \footnotetext[1]{Titty Peggy, Brudder Mosey, \&c.}
\end{song}

\begin{stanza}
\item[2.]
  And I had a mighty battle like-a Jacob and de angel, \\
  Jacob, time of old; \\
  I didn't 'tend to lef' 'em go \\
  Till Jesus bless my soul.

\item[3.]
  And bless\`ed me, and bless\`ed my, \\
  And bless\`ed all my soul; \\
  I didn't 'tend to lef' 'em go \\
  Till Jesus bless my soul.
\end{stanza}

\begin{extra}
  [This tune appears to be borrowed from ``And are ye sure the news is
    true?''---but it is so much changed, and the words are so
    characteristic, that it seemed undoubtedly best to retain it.]
\end{extra}


\newpage
\section{Hunting for a city.}
\thispagestyle{empty}

\begin{song}
  \lilypondfile[staffsize=18,line-width=356\pt]{024.ly}
\end{song}


\newpage
\section{Gwine follow.}
\thispagestyle{empty}

\begin{song}
  \lilypondfile[staffsize=18,line-width=356\pt]{025.ly}
\end{song}

\bigskip
\begin{extra}
  [The second part of this tune is evidently ``Buffalo'' (variously
    known also as ``Charleston'' or ``Baltimore'') ``Gals;'' the first
    part, however, is excellent and characteristic.]
\end{extra}


\newpage
\section{Lay this body down.}
\thispagestyle{empty}

\begin{song}
  \lilypondfile[staffsize=18,line-width=356\pt]{026.ly}
\end{song}

\begin{stanza}
\item[2.]
%% EDITED: Moved footnote to end of line.
  I know moonlight,\footnote[1]{O moonlight (\emph{or}
  moonrise); O my soul, O your soul.} I know starlight, \\
  I'm walkin' troo de starlight; \\
  Lay dis body down.
\end{stanza}

\begin{extra}
  [This is probably the song heard by W.~H.~Russell, of the London
    Times, as described in chapter xviii.~of ``My Diary North and
    South.''  The writer was on his way from Pocotaligo to
    Mr.~Trescot's estate on Barnwell Island, and of the midnight row
    thither he says:

    ``The oarsmen, as they bent to their task, beguiled the way by
    singing in unison a real negro melody, which was unlike the works
    of the Ethiopian Serenaders as anything in song could be unlike
    another.  It was a barbaric sort of madrigal, in which one singer
    beginning was followed by the others in unison, repeating the
    refrain in chorus, and full of quaint expression and
    melancholy:---

    `O your soul! oh my soul! I'm going to the churchyard \\
    To lay this body down; \\
    Oh my soul! oh your soul! we're going to the churchyard \\
    To lay this nigger down.'

    And then some appeal to the difficulty of passing the `Jawdam'
    constituted the whole of the song, which continued with unabated
    energy during the whole of the little voyage.  To me it was a
    strange scene.  The stream, dark as Lethe, flowing between the
    silent, houseless, rugged banks, lighted up near the landing by
    the fire in the woods, which reddened the sky---the wild strain,
    and the unearthly adjurations to the singers' souls, as though
    they were palpable, put me in mind of the fancied voyage across
    the Styx.''


    \newpage
    \thispagestyle{empty}

    We append with some hesitation the following as a variation; the
    words of which we borrow from Col.~Higginson.  Lt.~Col.~Trowbridge
    says of it that it was sung at funerals in the night time---one of
    the most solemn and characteristic of the customs of the negroes.
    He attributes its origin to St.~Simon's Island, Georgia:]

\medskip
\begin{song}
  \begin[staffsize=18,line-width=356\pt]{lilypond}
\score
{
  \new Staff {
    <<
    \context Voice
    {
	\set Staff.midiInstrument = "acoustic grand"
        \override Staff.VerticalAxisGroup #'minimum-Y-extent = #'(0 . 0)
	
	\autoBeamOff

	\time 4/4
	\clef violin
	\key g \minor

	     {
	       es''4 | es''4. es''8 es''4. es''8 |
	       d''4. d''8 d''4. d''8 |
	       c''4 c'' bes' a' | bes'2 r4 | \bar "||"
	     }
    }
    \lyricsto "" \new Lyrics
    {
      \override LyricText #'font-size = #0
      \override StanzaNumber #'font-size = #-1

      I know moon -- light, I know star -- light;
      I lay dis bo -- dy down.
    }
    >>
  }

  \layout { indent = 0.0 }
}
\end{lilypond}
\end{song}

\begin{stanza}
\item[2.]
  I walk in de moonlight, I walk in de starlight; \\
  I lay dis body down.

\item[3.]
  I know de graveyard, I know de graveyard, \\
  When I lay dis body down.

\item[4.]
  I walk in de graveyard, I wall troo de graveyard, \\
  To lay, \&c.

\item[5.]
  I lay in de grave an' stretch out my arms; \\
  I lay, \&c.

\item[6.]
  I go to de judgement in de evenin' of de day \\
  When I lay, \&c.

\item[7.]
  And my soul an' your soul will meet in de day \\
  When we lay, \&c.
\end{stanza}

\bigskip
    [``{}`I'll lie in de grave and stretch out my arms' Never, it
      seems to me, since man first lived and suffered, was his
      infinite longing for peace uttered more plaintively than in that
      line.''---\emph{Col.~Higginson.}]
\end{extra}


\newpage
\section{Heaven bell a-ring.}
\thispagestyle{empty}

\begin{song}
  \lilypondfile[staffsize=18,line-width=356\pt]{027.ly}
\end{song}

\begin{stanza}
\setlength{\itemsep}{1pt}
\item[2.]
  What shall I do for a hiding place?  And a heav'n, \&c.

\item[3.]
  I run to de sea, but de sea run dry.

\item[4.]
  I run to de gate, but de gate shut fast.

\item[5.]
  No hiding place for sinner dere.

\item[6.]
%% FIXME: Say when... instead Say you when...?
  Say you when you get to heaven say you 'member me.

\item[7.]
  Remember me, poor fallen soul.\footnotetext[1]{When I am gone, for
  Jesus' sake.}

\item[8.]
  Say when you get to heaven say your work shall prove.

\item[9.]
  Your righteous Lord shall prove 'em well.

\item[10.]
  Your righteous Lord shall find you out.

\item[11.]
  He cast out none dat come by faith.

\item[12.]
  You look to de Lord wid a tender heart.

\item[13.]
 I wonder where poor Monday dere.

\item[14.]
  For I am gone and sent to hell.

\item[15.]
  We must harkee what de worldy say.

\item[16.]
  Say Christmas come but once a year.

\item[17.]
  Say Sunday come but once a week.
\end{stanza}


\newpage
\section{Jine 'em.}
\thispagestyle{empty}

\begin{song}
  \lilypondfile[staffsize=18,line-width=356\pt]{028.ly}
\end{song}

\bigskip
\begin{extra}
  [For other words see ``Heaven bell a-ring,'' No.~27.  The following
    were sung at Hilton Head, probably to the same tune:

    %% FIXME: Indentation.
    Join, brethren, join us O, \\
    Join us, join us, O. \\
    We meet to-night to sing and pray; \\
    In Jesus' name we'll sing and pray.

    A favorite rowing tune: apparently a variation of ``Turn sinner,''
    No.~48.]
\end{extra}


\newpage
\section{Rain fall and wet Becca Lawton.}
\thispagestyle{empty}

\begin{song}
  \lilypondfile[staffsize=18,line-width=356\pt]{029.ly}

  \footnotetext[1]{Sun come and dry.}
  \footnotetext[2]{All de member, \&c.}
  \footnotetext[3]{We all, Believer, \&c.}
  \footnotetext[4]{Beat, Bent, Rack.}
\end{song}

\begin{stanza}
\item[2.]
  Do, Becca Lawton, come to me yonder.

\item[3.]
  Say, brudder Tony, what shall I do now?

\item[4.]
  Beat back holy, and rock salvation.
\end{stanza}

\begin{extra}
  [``Who,'' says Col.~Higginson, ``\emph{Becky Martin} was, and why
    she should or should not be wet, and whether the dryness was a
    reward or a penalty, none could say.  I got the impression that,
    in either case, the event was posthumous, and that there was some
    tradition of grass not growing over the grave of a sinner; but
    even this was vague, and all else vaguer.''

    Lt.~Col.~Trowbeidge heard a story that ``\emph{Peggy Norton} was
    an old prophetess, who said that it would not do to be baptised
    except when it rained; if the Lord was pleased with those who had
    been `in the wilderness,' he would sand rain.''  Mr.~Tomlinson
    says that the song always ends with a laugh, andd appears
    therefore to be regarded by the negroes as mere nonsense.  he adds
    that when it is used as a rowing tune, at the words ``Rack back
    holy!''\ one rower reaches over back and slaps the man behind him,
    who in turn does the same, and so on.]
\end{extra}


\newpage
\section{Bound to go.}
\thispagestyle{empty}

\begin{song}
  \lilypondfile[staffsize=18,line-width=356\pt]{030.ly}
\end{song}

\begin{stanza}
\item[2.]
  I build my house on shiftin' sand, \\
  De first wind come he blow him down.

\item[3.]
  I am not like de foolish man, \\
  He build his house upon de sand.

\item[4.]
  One mornin' as I was a walkin' along, \\
  I saw de berries a-hanging down.

\item[5.]
  I pick de berries and I suck de juice, \\
  He sweeter dan de honey comb.

%% FIXME: \item[6.]?
  I tuk dem brudder, two by two, \\
  I tuk dem sister, tree by tree.
\end{stanza}

\newpage
\thispagestyle{empty}

%%FIXME: center
\emph{Variation.}

\begin{song}
  \begin[staffsize=18,line-width=356\pt]{lilypond}
\score
{
  \new Staff {
    <<
    \context Voice
    {
	\set Staff.midiInstrument = "acoustic grand"
     \override Staff.VerticalAxisGroup #'minimum-Y-extent = #'(0 . 0)
	
	\autoBeamOff

	\time 2/4
	\clef violin
	\key d \major

	\override Stem #'neutral-direction = #1
	\partial 8 a'8 | d''8 d'' cis'' a' | b'8 b' a' r |
	d''4 e'' | a'4 r8 a'8 | d''8 d'' cis' a' | b'8 b' a' r |
	fis'4 e' | d'4 r8 \bar "||" \break

	r8 | a'8 b' a' a' | fis'8. fis'16 fis'8 r8 |
	a'8 << { \stemUp b'8 \stemNeutral }
	<< \\ { \stemDown a'8 } >> >> a'8 a' |
	e'8. e'16 e'8 r8 | a'8 b' a' a' | fis'8. g'16 a'8 r8 |
	d'16 d'8. fis'8. e'16 | d'4 r8 \bar "||"
    }
    \lyricsto "" \new Lyrics
    {
        \override LyricText #'font-size = #0
        \override StanzaNumber #'font-size = #-1

	I build my house up -- on a rock, O yes, Lord!
	No wind nor storm shall blow 'em down, O yes, Lord!
	March on, mem -- ber, Bound to go;
	March on, mem -- ber, Bound to go;
	March on, mem -- ber, Bound to go;
	Bid 'em fare you well.
    }
    >>
  }

  \layout { indent = 0.0 }
}
\end{lilypond}
\end{song}


\newpage
\section{Michael row the boat ashore.}
\thispagestyle{empty}

\begin{song}
  \lilypondfile[staffsize=18,line-width=356\pt]{031.ly}
\end{song}

\begin{stanza}
\setlength{\itemsep}{1pt}
\setlength{\parskip}{0pt}
\item[3.]
  I wonder where my mudder deh (there).

\item[4.]
  See my mudder on de rock gwine home.

\item[5.]
  On de rock gwine home in Jesus' name.

\item[6.]
  Michael boat a music boat.

\item[7.]
  Gabriel blow de trumpet horn.

\item[8.]
  O you mind your boastin' talk.

\item[9.]
  Boastin' talk will sink your soul.

\item[10.]
  Brudder, lend a helpin' hand.

\item[11.]
  Sister, help for trim dat boat.

\item[12.]
  Jordan stream is wide and deep.

\item[13.]
  Jesus stand on t' oder side.

\item[14.]
  I wonder if my maussa deh.

\item[15.]
  My fader gone to unknown land.

\item[16.]
  O de Lord he plant his garden deh.

\item[17.]
  He raise de fruit for you to eat.

\item[18.]
  He dat eat shall neber die.

\item[19.]
  When de riber overflow.

\item[20.]
  O poor sinner, how you land?

\item[21.]
  Riber run and darkness comin'.

\item[22.]
  Sinner row to save your soul.
\end{stanza}

%%FIXME: center
\emph{Words from Hilton Head.}

Michael haul the boat ashore. \\
Then you'll hear the horn they blow. \\
Then you'll hear the trumpet sound. \\
Trumpet sound the world around. \\
Trumpet sound for rich and poor. \\
Trumpet sound the jubilee. \\
Trumpet sound for you and me.


\newpage
\section{Sail, O believer.}
\thispagestyle{empty}

\begin{song}
  \lilypondfile[staffsize=18,line-width=356\pt]{032.ly}
\end{song}

\bigskip
\begin{extra}
  [Col.~Higginson gives the following stanzas, of which the above
    seems to be a part; but unfortunately he is unable to identify the
    music, which is well described by the terms in which he speaks of
    the words---''very graceful and lyrical, and with more variety of
    rhythm than usual:''

    %%FIXME: Indentation
    ``Bow low, Mary, bow low, Martha, \\
    For Jesus come and lock de door, \\
    And carry de keys away.

    Sail, sail, over yonder, \\
    And view de promised land, \\
    For Jesus come, \&c.

    Weep, O Mary, bow low, Martha, \\
    For Jesus come, \&c.

    Sail, sail, my true believer; \\
    Sail, sail, over yonder; \\
    Mary, bow low, Martha, bow low, \\
    For Jesus come and lock de door, \\
    And carry de keys away.'']
\end{extra}


\newpage
\section{Rock o' Jubilee.}
\thispagestyle{empty}

\begin{song}
  \lilypondfile[staffsize=18,line-width=356\pt]{033.ly}

  \footnotetext[1]{To mercy seat, To de corner o' de world.}
  \footnotetext[2]{Yes.}
\end{song}

\begin{stanza}
\item[3.]
  Stand back, Satan, let me come by.

\item[4.]
  O come, titty Katy, let me go.

\item[5.]
  I have no time for stay at home.

\item[6.]
  My Fader door wide open now.

\item[7.]
  Mary, girl, you know my name.

\item[8.]
  Look dis way an' you look dat way.

\item[9.]
  De wind blow East, he blow from Jesus.
\end{stanza}


\newpage
\section{Stars begin to fall.}
\thispagestyle{empty}

\begin{song}
  \lilypondfile[staffsize=18,line-width=356\pt]{034.ly}

  \footnotetext[1]{Titty Nelly, De member, \&c.}
\end{song}

\newpage
\section{King Emanuel.}
\thispagestyle{empty}

\begin{song}
  \lilypondfile[staffsize=18,line-width=356\pt]{035.ly}
\end{song}

\begin{stanza}
\item[3.]
  If you touch one string, den de whole heaven ring.

\item[4.]
  O the great cherubim, O de cherubim above.

\item[5.]
  O believer, ain't you glad dat your soul is converted?
\end{stanza}

\begin{extra}
  [This hymn---words and melody--bears all the marks of white origin.
    We have not, however, been able to find it in any hymn-book, and
    therefore retain it, as being a favorite at Port Royal.]
\end{extra}


\newpage
\section{Satan's Camp A-Fire.}
\thispagestyle{empty}

\begin{song}
  \lilypondfile[staffsize=18,line-width=356\pt]{036.ly}
\end{song}


\newpage
\section{Give up the world.}
\thispagestyle{empty}

\begin{song}
  \lilypondfile[staffsize=18,line-width=356\pt]{037.ly}

  \footnotetext[1]{De moon give a light, De starry crown.}
\end{song}

\bigskip
\begin{extra}
  [The first movement of this air is often sung in the minor key.]
\end{extra}


\newpage
\section{Jesus on the Waterside.}
\thispagestyle{empty}

\begin{song}
  \lilypondfile[staffsize=18,line-width=356\pt]{038.ly}
\end{song}


\newpage
\section{I wish I been dere.}
\thispagestyle{empty}

\begin{song}
  \lilypondfile[staffsize=18,line-width=356\pt]{039.ly}
\end{song}


\newpage
\section{Build a house in Paradise.}
\thispagestyle{empty}

\begin{song}
  \lilypondfile[staffsize=18,line-width=356\pt]{040.ly}
\end{song}


\newpage
\section{I know when I'm going home.}
\thispagestyle{empty}

\begin{song}
  \lilypondfile[staffsize=18,line-width=356\pt]{041.ly}
\end{song}


\newpage
\section{I'm a trouble in de mind.}
\thispagestyle{empty}

\begin{song}
  \lilypondfile[staffsize=18,line-width=356\pt]{042.ly}

  \footnotetext[1]{Titty Rosy, Brudder Johnny, Come along dere.}
\end{song}

\newpage
\section{Travel on.}
\thispagestyle{empty}

\begin{song}
  \lilypondfile[staffsize=18,line-width=356\pt]{043.ly}

  \footnotetext[1]{Heaven-bound.}
\end{song}

\newpage
%% CHANGED: Add comma as in TOC.
\section{Archangel, open the door.}
\thispagestyle{empty}

\begin{song}
  \lilypondfile[staffsize=18,line-width=356\pt]{044.ly}

  \footnotetext[1]{Sister.}
\end{song}

\begin{stanza}
\item[2.]
  Brudder, tuk off your knapsack, I'm gwine home;\\
  Archangel open de door.
\end{stanza}


\newpage
\section{My body rock 'long fever.}
\thispagestyle{empty}

\begin{song}
  \lilypondfile[staffsize=18,line-width=356\pt]{045.ly}

  \footnotetext[1]{All de member.}
  \footnotetext[2]{Long time seeker 'gin to believe.}
\end{song}

\begin{stanza}
\item[2.]
  By de help ob de Lord we rise up again\\
  O de Lord he comfort de sinner;\\
  By de help ob de Lord we rise up again\\
  An' we'll get to heaven at last.
\end{stanza}

\newpage
\thispagestyle{empty}

%%FIXME: center
\emph{Variation.}

\begin{song}
  \begin[staffsize=18,line-width=356\pt]{lilypond}
\score
{
  \new Staff {
    <<
    \context Voice
    {
	\set Staff.midiInstrument = "acoustic grand"
	\override Staff.VerticalAxisGroup #'minimum-Y-extent = #'(0 . 0)
	
	\autoBeamOff

	\time 2/4
	\clef violin
	\key d \minor

	\partial 8 f'16 f' | g'8 g' g' f'16 g' | bes'8 bes' bes' d' |
	f'8 f' d'16 f'16 | g'8 f' d' f' | g'8 g' g' f'16 g' |
	bes'8 bes' bes' c'' | d''16 d'' d'' c'' bes'8 bes'16 a'16 |
	g'4( g'8) \bar "||"
    }
    \lyricsto "" \new Lyrics
    {
      \override LyricText #'font-size = #0
      \override StanzaNumber #'font-size = #-1

      O my bo -- dy's racked wid de fe -- ve -- er,
      My head rack'd wid de pain I hab,
      I wish I was in de king -- do -- om,
      A -- set -- tin' on de side ob de Lord.
    }
    >>
  }

  \layout { indent = 0.0 }
}
\end{lilypond}
\end{song}


\bigskip
[This is one of the most striking of the Port Royal spirituals, and is
shown by a comparison with No.~93 to be one of the most widely spread
of all the African hymns.  It is hard to explain every word of the
introduction, but ``long time get over crosses'' is of course the
``long time waggin' o' er de crossin'{}'' of the Virginia hymn.]
    
  
\newpage
\section{Bell da ring.}
\thispagestyle{empty}

\begin{song}
  \lilypondfile[staffsize=18,line-width=356\pt]{046.ly}

  \footnotetext[1]{Boggy, Tedious.}
  \footnotetext[2]{'ciety, Lecter, Praise-house.}
\end{song}

\begin{stanza}
\item[3.]
  I can't get to meetin'.\footnotemark[2]
\item[4.]
  De church mos' ober.
\item[5.]
  De heaven-bell a heaven-bell.
\item[6.]
  De heaven-bell I gwine home.
\item[7.]
  I shout for de heaven-bell.
\item[8.]
  Heaven 'nough for me one.
\item[9.]
  (Brudder) hain't you a member.
\end{stanza}

\newpage
\thispagestyle{empty}

[The following words were sung in Col.~Higginson's regiment:

Do my brudder, O yes, yes, member,\\
De bell done ring.\\
You can't get to heaben\\
When de bell done ring.\\
If you want to get to heaven,\\
Fo' de bell, etc.\\
You had better follow Jesus,\\
Fo' de bell, etc.\\
O yes, my Jesus, yes, I member,\\
De bell etc.\\
O come in, Christians,\\
Fo' de bell, etc.\\
For the gates are all shut,\\
When de bell, etc.\\
And you can't get to heaben\\
When de bell, etc.\\

Col.~Higginson suggests that this refrain may have originated in
Virginia, and gone South with our army, because ``{}`done' is a
Virginia shibboleth, quite distinct from the `been' which replaces it
in South Carolina.  In the proper South Carolina dialect, would have
been substituted `De bell been-a ring.'{}'' We have, however, shown in
the preface, that ``done'' is used on St.~Helena; and at any rate the
very general use of this refrain there in the present tense, ``Bell da
ring,'' would indicate that it was of local origin, while we have
never met with anything at all like it in any other part of the
country.  As given above, it is one of the most characteristic
``shouting'' tunes.

In singing ``Heaven-bell a heaven-bell,'' the \emph{v} and \emph{n}
were so run together that the words sounded like ``hum-bell a
hum-bell,'' with strong emphasis and dwelling upon the \emph{m}.]


\newpage
\section{Pray all de member.}
\thispagestyle{empty}

\begin{song}
  \lilypondfile[staffsize=18,line-width=356\pt]{047.ly}

  \footnotetext[1]{True believer.}
\end{song}

\begin{stanza}
\item[4.]
  Jericho, Jericho.
\item[5.]
  I been to Jerusalem.
\item[6.]
  Patrol aroun'me.
\item[7.]
  Thank God he no ketch me.
\item[8.]
  Went to de meetin'
\item[9.]
  Met brudder Hacless [Hercules].
\item[10.]
  Wha' d'ye tink he tell me?
\item[11.]
  Tell me for to turn back.
\item[12.]
  Jump along Jericho.
\end{stanza}

\begin{extra}
  [This also is a very characteristic shouting tune.]
\end{extra}

\newpage
\section{Turn sinner, turn O.}
\thispagestyle{empty}

\begin{songpart}
  \lilypondfile[staffsize=18,line-width=356\pt]{048-p1.ly}
\end{songpart}
\begin{songpart}
  \lilypondfile[staffsize=18,line-width=356\pt]{048-p2.ly}
\end{songpart}
\begin{songpart}
  \lilypondfile[staffsize=18,line-width=356\pt]{048-p3.ly}
\end{songpart}

\newpage
\thispagestyle{empty}

\begin{songpart}
  \lilypondfile[staffsize=18,line-width=356\pt]{048-p3b.ly}
\end{songpart}
\begin{song}
  \lilypondfile[staffsize=18,line-width=356\pt]{048-p4.ly}
\end{song}

\bigskip

The following words are sung to the same tune:

\begin{stanza}
\setlength{\itemsep}{1pt}
\setlength{\parskip}{0pt}
\item[1.]
  Bro' Joe, you ought to know my name---Hallelujah.
\item[2.]
  My name is written in de book ob life.
\item[3.]
  If you look in de book you'll fin' 'em dar.
\item[4.]
  One mornin' I was a walkin' down.
\item[5.]
  I saw de berry a-hinging down.
\item[6.]
  (Lord) I pick de berry, an' I suck de juice.
\item[7.]
  Jes' as sweet as de honey in de comb.
\item[8.]
  I wonder where fader Jimmy gone.
\item[9.]
  My fader gone to de yonder worl'.
\item[10.]
  You dig de spring dat nebber dry.
\item[11.]
  De more I dig 'em, de water spring.
\item[12.]
  De water spring dat nebber dry.
\end{stanza}

\begin{extra}
  [This is the most dramatic of all the shouts; the tune varies with the
    words, commonly about as given above, and the general effect is very
    pathetic.  The words and tunes are constantly interchanged: thus, for
    instance, the 6th versemight be sung to the second variation, and the
    8th, 9th, and 10th, to the third.]
\end{extra}

\newpage
\section{My army cross over.}
\thispagestyle{empty}

\begin{song}
  \lilypondfile[staffsize=18,line-width=356\pt]{049.ly}
\end{song}

%% FIXME: Table in two columns?
\begin{stanza}
\item[2.]
  Satan bery busy.
\item[3.]
  Wash 'e face in ashes.
\item[4.]
  Put on de leder apron.
\item[5.]
  Jordan riber rollin'.
\item[6.]
  Cross 'em, I tell ye, cross 'em.
\item[7.]
  Cross Jordan (danger) riber.
\end{stanza}

\begin{extra}
  [The following version, probably from Sapelo Id., Georgia, was sung in
    Col.~Higginson's regiment:
\end{extra}

\begin{song}
  \begin[staffsize=18,line-width=356\pt]{lilypond}
\score
{
  \new Staff {
    <<
    \context Voice
    {
	\set Staff.midiInstrument = "acoustic grand"
     \override Staff.VerticalAxisGroup #'minimum-Y-extent = #'(0 . 0)
	
	\autoBeamOff

	\time 4/4
	\clef violin
	\key g \major

	\partial 4 g'4 | g'4 g'4 b'2 | a'4 a' r fis' | fis'4 fis' d'2 |
	g'4 g' r g' | b'4 b' g' g' | fis'4 a' r fis' | fis'4 fis' d'2 |
	g'4 g' r g' | fis' 4 a' r g' | fis'4 a' r g' | fis'4 fis' d'2 |
	g'4 g' r \bar "||"
    }
    \lyricsto "" \new Lyrics
    {
      \override LyricText #'font-size = #0
      \override StanzaNumber #'font-size = #-1

      \set stanza = "1."
      My ar -- my cross o -- ber,
      My ar -- my cross o -- ber,
      O Phara -- oh's ar -- my drown -- ded,
      My ar -- my cross o -- ber.
      My ar -- my, my ar -- my,
      my ar -- my cross o -- ber.
    }
    >>
  }

  \layout { indent = 0.0 }
}
\end{lilypond}
\end{song}

\begin{stanza}
\item[2.]
  We'll cross de riber Jordan.
\item[3.]
  We'll cross de danger water.
\item[4.]
  We'll cross de mighty Myo.
\end{stanza}

\begin{extra}
[On the word ``Myo,'' Col.~Higginson makes the following note: ``I
could get no explanation of the `mighty Myo,' except that one of the
old men thought it meant the river of death.  Perhaps it is an African
word.  In the Cameroon dialect, `Mawa' signifies `to die.'{}''
Lt.~Col.~Trowbridge feels very confident that it is merely corruption
of ``bayou.'']
\end{extra}

\newpage
\section{Join the angel band.}
\thispagestyle{empty}

\begin{song}
  \lilypondfile[staffsize=18,line-width=356\pt]{050.ly}
\end{song}

%% FIXME: Table in two columns?
\begin{stanza}
\item[2.]
  Do, fader Modey, gader your army.
\item[3.]
  O do mo' soul gader togeder.
\item[4.]
  O do join 'em, join 'em for Jesus.
\item[5.]
  O do join 'em, join 'em archangel.
\end{stanza}


\begin{extra}
  The following variation of the first line, with the words that follow,
  was sung in Charleston:
\end{extra}

\begin{song}
  \begin[staffsize=18,line-width=356\pt,ragged-right]{lilypond}
\score
{
  \new Staff {
    <<
    \context Voice
    {
	\set Staff.midiInstrument = "acoustic grand"
     \override Staff.VerticalAxisGroup #'minimum-Y-extent = #'(0 . 0)
	
	\autoBeamOff

	\time 2/4
	\clef violin
	\key a \major

	\partial 4 e''4 | cis''8 e'' e''4 | cis''8. a'16 fis'8 a'
    }
    \lyricsto "" \new Lyrics
    {
      \override LyricText #'font-size = #0
      \override StanzaNumber #'font-size = #-1

      O join 'em all, join for Jesus.
    }
    >>
  }

  \layout { indent = 0.0 }
}
\end{lilypond}
\end{song}

\begin{stanza}
\item[]
  O join 'em all, join for Jesus, Join Jerusalem Band.
\item[]
  Sister Mary, stan' up for Jesus.
\item[]
  O brudder an' sister, come up for Heaven.
\item[]
  Daddy Peter set out for Jesus.
\item[]
  Ole Maum Nancy set out for Heaven.
\end{stanza}

\begin{extra}
  [``The South Carolina negroes never say Aunty and Uncle to old
    persons, but Daddy and Maumer, and all the white people say Daddy and
    Maumer to old black men and women''---A.~M.~B.

    This is no doubt correct as regards South Carolina in general.  I am
    sure that I heard ``Uncle'' and ``Aunty'' at Port Royal, and I do not
    remember hearing ``Daddy'' and ``Maumer'.''---W.~F.~A.]
\end{extra}


\newpage
\section{I an' Satan had a race.}
\thispagestyle{empty}

\begin{song}
  \lilypondfile[staffsize=18,line-width=356\pt]{051.ly}
\end{song}

\begin{stanza}
\item[2.]
  Win de race agin de course.
\item[3.]
  Satan tell me to my face
\item[4.]
  He will break my kingdom down.
\item[5.]
  Jesus whisper in my heart
\item[6.]
  He will build 'em up again.
\item[7.]
  Satan mount de iron grey;
\item[8.]
  Ride half way to Pilot-Bar.
\item[9.]
  Jesus mount de milk-white horse.
\item[10.]
  Say you cheat my fader children.
\item[11.]
  Say you cheat 'em out of glory.
\item[12.]
  Trouble like a gloomy cloud
\item[13.]
  Gader dick an' tunder loud.
\end{stanza}


\newpage
\section{Shall I die?}
\thispagestyle{empty}

\begin{song}
  \lilypondfile[staffsize=18,line-width=356\pt]{052.ly}
\end{song}

\begin{stanza}
\item[3.]
  Die, die die, shall I die? \\
  Jesus da coming, shall I die?

\item[4.]
  Run for to meet him, shall I die? \\
  Weep like a weeper, shall I die?

\item[5.]
  Mourn like a mourner, shall I die? \\
  Cry like a crier, shall I die?
\end{stanza}

\begin{extra}
  [This shout was a great favorite of the Capt.~John Fripp plantation;
    its simplicity, wildness and minor character suggest a native African
    origin.  Sometimes the leading singer would simply repeat the words,
    mornfully: ``Die, die, die,''---sometimes he would interpolate such an
    inappropriate line as ``Jump along, jump along dere.'']
\end{extra}


\newpage
\section{When we do meet again.}
\thispagestyle{empty}

\begin{song}
  \lilypondfile[staffsize=18,line-width=356\pt]{053.ly}
\end{song}


\newpage
\section{The White Marble Stone.}
\thispagestyle{empty}

\begin{song}
  \lilypondfile[staffsize=18,line-width=356\pt]{054.ly}

  %% We have to provide the text of the footnotes in this document.
  \footnotetext[1]{Believer, Patty, etc.}
\end{song}

\begin{stanza}
\item[2.]
  O the city light the lamp, the white man he will sold,\\
  And I wish I been there, etc.
\item[3.]
  O the white marble stone, and the white marble stone.
\end{stanza}

\begin{extra}
  [This song was described to us as ``\emph{too} pretty.''  The
    following minor variation might be called ``too \emph{much}
    prettier.''
\end{extra}

\begin{song}
  \begin[staffsize=18,line-width=356\pt]{lilypond}
\score
{
  \new Staff {
    <<
    \context Voice
    {
	\set Staff.midiInstrument = "acoustic grand"
     \override Staff.VerticalAxisGroup #'minimum-Y-extent = #'(0 . 0)
	
	\autoBeamOff

	\time 6/8
	\clef violin
	\key d \minor

	\partial 8 d'16 f' | f'16 f' f'8. f'16 f'8 bes' a' |
	g'8 a'8. gis'16 a'4 r16 f' |
	f'8 a' c'' d'' d'8. e'16 | f'16 f' e'8. f'16 d'8 r \bar "||"
    }
    \lyricsto "" \new Lyrics
    {
      \override LyricText #'font-size = #0
      \override StanzaNumber #'font-size = #-1

      O my sis -- ter light de lamp, and de lamp light de road;
      I wish I been dere for to hear -- de Jor -- dan roll.
    }
    >>
  }

  \layout { indent = 0.0 }
}
\end{lilypond}
\end{song}


\newpage
\section{I can't stand the fire.}
\thispagestyle{empty}

\begin{song}
  \lilypondfile[staffsize=18,line-width=356\pt]{055.ly}
\end{song}

\begin{extra}
  [Probably only a fragment of a longer piece.  The following variation
    was sung at Coffin's Point:
\end{extra}

\begin{song}
  \begin[staffsize=18,line-width=356\pt,ragged-right]{lilypond}
\score
{
  \new Staff {
    <<
    \context Voice
    {
	\set Staff.midiInstrument = "acoustic grand"
     \override Staff.VerticalAxisGroup #'minimum-Y-extent = #'(0 . 0)
	
	\autoBeamOff

	\time 2/8
	\clef violin
	\key f \major

	g'8 g'16. a'32 | f'8 r |
	g'8 g'16. a'32 | f'8 r |
	\break
	
	bes'8 bes'16. bes'32 |
	<< { \once \override Stem #'transparent = ##t a'16 
	  \set fontSize = #'-4 \stemDown c''16 a' f'
	  \set fontSize = #'0 \stemNeutral }
	<< \\ \set fontSize = #'0 \stemUp { a'8 r8 } >> >> |
	g'8 g'16. g'32. | f'8 r	\bar "||"
    }
    \lyricsto "" \new Lyrics
    {
      \override LyricText #'font-size = #0
      \override StanzaNumber #'font-size = #-1

      Can't stand the fire,
      Can't stand the fire,
      Can't stand the fire, "(O" Lord, "I)"
      Can't stand the fire.
    }
    >>
  }

  \layout { indent = 0.0 }
}
\end{lilypond}
\end{song}


\newpage
\section{Meet, O Lord.}
\thispagestyle{empty}

\begin{song}
  \lilypondfile[staffsize=18,line-width=356\pt]{056.ly}

  %% We have to provide the text of the footnotes in this document.
  \footnotetext[1]{\emph{i.~e.} the anointing vial.}
\end{song}

\begin{stanza}
\item[2.]
  Moon went into de poplar tree,\\
  An' star went into blood;\\
  In dat mornin', etc.
\end{stanza}

\begin{extra}
  [This was taught me by a boy from Hilton Head Island, whom the rebel
    Gen.~Drayton left holding his horse ``when gun shoot at Bay Pint.''
    The General never returned to reclaim his horse, which afterwards came
    into the possession of a friend of mine, and was famed for swiftness.
    I had several fine rides upon ``milk-white''
    \emph{Drayton}.---W.~F.~A.]
\end{extra}


\newpage
\section{Wai', Mr.~Mackright}
\thispagestyle{empty}

\begin{song}
  \lilypondfile[staffsize=18,line-width=356\pt]{057.ly}
\end{song}


\newpage
\section{Early in the morning.}
\thispagestyle{empty}

\begin{song}
  \lilypondfile[staffsize=18,line-width=356\pt]{058.ly}

  \footnotetext[1]{O shout glory till 'em join dat ban.}
\end{song}

\begin{stanza}
\setlength{\itemsep}{1pt}
\setlength{\parskip}{0pt}
\item[2.]
  I meet my mudder early in de mornin',\\
  An' I ax her, how you do my mudder?\\
  Walk 'em easy, etc.
\item[3.]
  I meet brudder Robert early in de mornin';\\
  I ax brudder Robert, how you do, my sonny?
\item[4.]
  I meet titta-Wisa\footnote[2]{\emph{i.~e.} sister Louisa.}
  early in de mornin';\\
  I ax titta-Wisa, how you do, my darter?
\end{stanza}

\emph{Variation of first part.}

\vspace{-10pt}

\begin{song}
  \begin[staffsize=18,line-width=356\pt]{lilypond}
\score
{
  \new Staff {
    <<
    \context Voice
    {
	\set Staff.midiInstrument = "acoustic grand"
     \override Staff.VerticalAxisGroup #'minimum-Y-extent = #'(0 . 0)
	
	\autoBeamOff

	\time 2/4
	\clef violin
	\key g \major
	{
	    \repeat volta 2
	    {
		\partial 8 c''8 | c''8 c''16 c'' b'8 b' |
		a'16 a' a' g' e'8 g' | c''4 b' a'8. g'16 e'4 |
		g'16 g' g' e' d'4 fis'16 r
	    }
	}
    }
    \lyricsto "" \new Lyrics
    {
      \override LyricText #'font-size = #0
      \override StanzaNumber #'font-size = #-1

      \set stanza = "1."
      I meet lit -- tle Ro -- sa ear -- ly in de morn -- in',
      O Je -- ru -- sa -- lem, ear -- ly in de morn -- "in'; etc."
    }
    >>
  }

  \layout { indent = 0.0 }
}
\end{lilypond}
\end{song}


\newpage
\section{Hail, Mary.}
\thispagestyle{empty}

\begin{song}
  \lilypondfile[staffsize=18,line-width=356\pt]{059.ly}
\end{song}

\begin{extra}
  [``I fancied,'' says Col.~Higginson, ``that the original reading
    might have been `soul,' instead of `soldier,'---with some other
    syllable inserted, to fillout the metre,---and that the `Hail,
    Mary,' might denote a Roman Catholic origin, as I had several men
    from St.~Augustine who held in a dim way to that faith.''

    In Mr.~Spaulding's article in the \emph{Continental Monthly,} a
    tune nearly identical with this is given with words almost the
    same as those of ``No more peck of corn,'' No.~64, the whole as an
    introduction to the second part of ``Trouble of the World,''
    No.~10---a curious illustration of the way in which the colored
    people make different combinations of their own tunes at different
    times:]
\end{extra}

\begin{song}
  \begin[staffsize=18,line-width=356\pt]{lilypond}
\score
{
  \new Staff {
    <<
    \context Voice
    {
	\set Staff.midiInstrument = "acoustic grand"
        \override Staff.VerticalAxisGroup #'minimum-Y-extent = #'(0 . 0)
	
	\autoBeamOff

	\time 2/4
	\clef violin
	\key g \major

	d'8 g' g' g' | g'8 g'4. | e'8 e' g' g' | d'8 d'4. |
	d'8 fis' a' fis' | g'8[ c''] c''4 | b'4 a'8 a' | g'4 r \bar "||"
    }
    \lyricsto "" \new Lyrics
    {
      \override LyricText #'font-size = #0
      \override StanzaNumber #'font-size = #-1

      \set stanza = "1."
      Done wid dri -- ber's dri -- bin',
      Done wid dri -- ber's dri -- bin',
      Done wid dri -- ber's dri -- bin',
      Roll, Jor -- dan, roll.
    }
    >>
  }

  \layout { indent = 0.0 }
}
\end{lilypond}
\end{song}

\begin{stanza}
\item[2.]
  Done wid massa's hollerin',
\item[3.]
  Done wid missus' scoldin'.
\end{stanza}


\newpage
\section{No more rain fall for wet you.}
\thispagestyle{empty}

\begin{song}
  \lilypondfile[staffsize=18,line-width=356\pt]{060.ly}
\end{song}

\begin{stanza}
\item[2.]
  No more sun shine for burn you.
\item[3.]
  No more parting in de kingdom.
\item[4.]
  No more backbiting in de kingdom.
\item[5.]
  Every day shall be Sunday.
\end{stanza}


\newpage
\section{I want to go home.}
\thispagestyle{empty}

\emph{In chanting style.}

\begin{song}
  \lilypondfile[staffsize=18,line-width=356\pt]{061.ly}
\end{song}

\begin{stanza}
\item[2.]
  Dere's no sun to burn you,---O yes, etc.
\item[3.]
  Dere's no hard trials.
\item[4.]
  Dere's no whips a--crackin'.
\item[5.]
  Dere's no stormy weather.
\item[6.]
  Dere's no tribulation.
\item[7.]
  No more slavery in de kingdom.
\item[8.]
  No evil--doers in de kingdom.
\item[9.]
  All is gladness in de kingdom.
\end{stanza}


\newpage
\section{Good-bye, brother.}
\thispagestyle{empty}

\begin{song}
  \lilypondfile[staffsize=18,line-width=356\pt]{062.ly}
\end{song}

\begin{stanza}
\item[2.]
  We part in de body but we meet in de spirit,\\
  We'll meet in de heaben in de blessed\footnote[1]{Glorious.} kingdom.
\item[3.]
  So good--bye, brother, good--bye, sister;\\
  Now God bless you, now God bless you.
\end{stanza}

\begin{extra}
  [``Sung at the breaking up of a midnight meeting after the death of
    a soldier.  These midnight \emph{wails} are very solemn to me, and
    exhibit the sadness of the present mingled with the joyful hope of
    the future.  I have known the negroes to get together in groups of
    six or eight around a small fire, and sing and pray alternately
    from nine o'clock till three the next morning, after the death of
    one of their number.''---J.~S.~R.]
\end{extra}


\newpage
\section{Fare ye well.}
\thispagestyle{empty}

\begin{song}
  \lilypondfile[staffsize=18,line-width=356\pt]{063.ly}
\end{song}


\newpage
\section{Many thousand go.}
\thispagestyle{empty}

\begin{song}
  \lilypondfile[staffsize=18,line-width=356\pt]{064.ly}
\end{song}

\begin{stanza}
\item[2.]
  No more driver's lash for me.
\item[3.]
  No more pint o' salt for me.
\item[4.]
  No more hundred lash for me.
\item[5.]
  No more mistress' call for me.
\end{stanza}

\begin{extra}
  [A song ``to which the Rebellion had actually given rise.  This was
    composed by nobody knows whom---though it was the most recent
    doubtless of all these `spirituals,'---and had been sung in secret
    to avoid detection.  It is certainly plaintive enough.  The peck
    of corn and pint of salt were slavery's rations.''---T.~W.~H.
    Lt.~Col.~Trowbridge learned that it was first sung when Beauregard
    took the slaves of the islands to build the fortifications at
    Hilton Head and Bay Point.]
\end{extra}


\newpage
\section{Brother Moses gone.}
\thispagestyle{empty}

\begin{song}
  \lilypondfile[staffsize=18,line-width=356\pt]{065.ly}
\end{song}


\newpage
\section{The Sin-sick Soul.}
\thispagestyle{empty}

\begin{song}
  \lilypondfile[staffsize=18,line-width=356\pt]{066.ly}
\end{song}


\newpage
\section{Some Valiant Soldier.}
\thispagestyle{empty}

\begin{song}
  \lilypondfile[staffsize=18,line-width=356\pt]{067.ly}
\end{song}

\bigskip
[The words are in part the same as those of ``Hail Mary,'' No.~59.]


\newpage
\section{Hallelu, Hallelu.}
\thispagestyle{empty}

\begin{song}
  \lilypondfile[staffsize=18,line-width=356\pt]{068.ly}
\end{song}

\begin{stanza}
\item[3.]
  Member walk and never tire.
\item[4.]
  Member walk Jordan long road.
\item[5.]
  Member walk tribulation.
\item[6.]
  You go home to Wappoo.
\item[7.]
  Member seek new repentance.
\item[8.]
  I go to seek my fortune.
\item[9.]
  I go to sek my dying Saviour.
\item[10.]
  You want to die like Jesus.
\end{stanza}

\begin{extra}
  [For other words, see ``Children do linger,'' No.~69.]
\end{extra}

\newpage
\section{Children do linger.}
\thispagestyle{empty}

\begin{song}
  \lilypondfile[staffsize=18,line-width=356\pt]{069.ly}
\end{song}

\begin{stanza}
\item[3.]
  O Jesus is our Captain.
\item[4.]
  He lead us on to glory.
\item[5.]
  We'll meet at Zion gateway.\footnote[1]{Heaven portal.}
\item[6.]
  We'll take dis story over.
\item[7.]
  We'll enter into glory.
\item[8.]
  When we done wid dis world trials.
\item[9.]
  We done wid all our crosses.
\item[10.]
  O brudder, will you meet us?
\item[11.]
  When de ships is out a--sailin'.
\item[12.]
  O Jesus got de hellum.
\item[13.]
  Fader, gader in your chil'en.
\item[14.]
  O gader dem for Zion.
\item[15.]
  'Twas a beauteous Sunday mornin'.
\item[16.]
  When he rose from de dead.
\item[17.]
  He will bring you milk and honey.
\end{stanza}


\newpage
\section{Good-bye.}
\thispagestyle{empty}

\begin{song}
  \lilypondfile[staffsize=18,line-width=356\pt]{070.ly}
\end{song}

\bigskip
\begin{extra}
  [``This is sung at the breaking up of a meeting, with a general
    shaking of hands, and the name of him or her pronounced, whose
    hand is shaken; of course there is seeming
    confusion.''---Mrs.~C.~J.~B.]
\end{extra}


\newpage
\section{Lord, make me more patient.}
\thispagestyle{empty}

\begin{song}
  \lilypondfile[staffsize=18,line-width=356\pt]{071.ly}

  \footnotetext[1]{``Any adjective expressive of the virtues is inserted
    here: holy, loving, peaceful, etc.''---Mrs.~C.~J.~B.}
\end{song}

\newpage
\section{The Day of Judgment.}
\thispagestyle{empty}

\begin{song}
  \lilypondfile[staffsize=18,line-width=356\pt]{072.ly}

  \footnotetext[1]{``A sort of prolonged wail.''---Mrs.~C.~J.~B.}
\end{song}


\begin{stanza}
\item[2.]
  And you'll see de stars a-fallin'.
\item[3.]
  And de world will be on fire.
\item[4.]
  And you'll hear de saints a-singin:
\item[5.]
  And de Lord will say to de sheep.
\item[6.]
  For to go to Him right hand;
\item[7.]
  But de goats must go to de left.
\end{stanza}


\newpage
\section{The Resurrection Morn.}
\thispagestyle{empty}

\begin{song}
  \lilypondfile[staffsize=18,line-width=356\pt]{073.ly}
\end{song}

\begin{stanza}
\item[3.]
  That she went to de sepulchre.
\item[4.]
  And de Lord he wasn't da.
\item[5.]
  But she see a man a--comin'.
\item[6.]
  And she thought it was de gardener.
\item[7.]
  But he say, ``O touch me not',
\item[8.]
  ``For I am not yet ascended.
\item[9.]
  ``But tell to my disciples
\item[10.]
  ``Dat de Lord he is arisen.''
\item[11.]
  So run, Mary, run, etc.
\end{stanza}


\newpage
\section{Nobody knows the trouble I've had.}
\thispagestyle{empty}

\begin{song}
  \lilypondfile[staffsize=18,line-width=356\pt]{074.ly}

  \footnotetext[1]{I see.}
\end{song}

\begin{stanza}
\item[2.]
  I pick de berry and I suck de juice, O yes, Lord!\\
  Just as sweet as the honey in de comb, O yes, Lord!
\item[3.]
  Sometimes I'm up, sometimes I'm down,\\
  Sometimes I'm almost on de groun'.
\item[4.]
  What make ole Satan hate me so?\\
  Because he got me once and he let me go.
\end{stanza}

\emph{Variation on St.~Helena Id.}

\begin{song}
  \begin[staffsize=18,line-width=356\pt,ragged-right]{lilypond}
\score
{
  \new Staff {
    <<
    \context Voice
    {
	\set Staff.midiInstrument = "acoustic grand"
     \override Staff.VerticalAxisGroup #'minimum-Y-extent = #'(0 . 0)
	
	\autoBeamOff

	\time 4/4
	\clef violin
	\key bes \major

	f''2 f'' | d''2 r4 d'' | f''4 f'' d'' bes' | f''4 d'' bes' r
    }
    \lyricsto "" \new Lyrics
    {
      \override LyricText #'font-size = #0
      \override StanzaNumber #'font-size = #-1

      O yes, Lord!
      I saw some ber -- ries hang -- ing down.
    }
    >>
  }

  \layout { indent = 0.0 }
}
\end{lilypond}
\end{song}


\newpage
\thispagestyle{empty}

\begin{extra}
  [This song was a favorite in the colored schools of Charleston in
    1865; it has since that time spread to the Sea Islands, where it
    is now sung with the variation noted above.  An independent
    transcription of this melody, sent from Florida by
    Lt.~Col.~Apthorp, differed only in the ictus of certain measures,
    as has also been noted above.  The third verse was furnished by
    Lt.~Col.~Apthorp.  Once when there had been a good deal of ill
    feeling excited, and trouble was apprehended, owing to the
    uncertain action of Government in regard to the confiscated lands
    on the Sea Islands, Gen.~Howard was called upon to address the
    colored people earnestly and even severly.  Sympathizing with
    them, however, he could not speak to his own satisfaction; and to
    relieve their minds of the ever-present sense of injustice, and
    prepare them to listen, he asked them to sing.  Immediately an old
    woman on the outskirts of the meeting began ``Nobody knows the
    trouble I've had','' and the whole audience joined in.  The
    General was so affected by the plaintive words and melody, that he
    found himself melting into tears and quite unable to maintain his
    official sterness.]
\end{extra}


\newpage
\section{Who is on the Lord's side.}
\thispagestyle{empty}

\begin{song}
  \lilypondfile[staffsize=18,line-width=356\pt]{075.ly}
\end{song}

\begin{stanza}
\item[2.]
  Weepin' Mary.
\item[3.]
  Mournin' Marta.
\item[4.]
  Risen Jesus.
\end{stanza}


\newpage
\section{Hold out to the end.}
\thispagestyle{empty}

\begin{song}
  \lilypondfile[staffsize=18,line-width=356\pt]{076.ly}
\end{song}


\newpage
\section{Come go with me.}
\thispagestyle{empty}

\begin{song}
  \lilypondfile[staffsize=18,line-width=356\pt]{077.ly}
\end{song}

\begin{stanza}
\item[2.]
  I did not come here myself, my Lord,\\
  It was my Lord who brought me here;\\
  And I really do believe I'm a child of God,\\
  A--walkin' in de heaven I roam.\\
  O come--e go wid me, etc.
\end{stanza}


\newpage
\section{Every hour in the day.}
\thispagestyle{empty}

\begin{song}
  \lilypondfile[staffsize=18,line-width=356\pt]{078.ly}
\end{song}

\begin{stanza}
\item[2.]
  Every hour in de night cry Jesus, etc.
\end{stanza}


\newpage
\section{In the mansions above.}
\thispagestyle{empty}

\begin{song}
  \lilypondfile[staffsize=18,line-width=356\pt]{079.ly}
\end{song}

\begin{stanza}
\item[2.]
  My Lord, I've had many crosses an' trials here below;\\
  My Lord, I hope to meet you\\
  In de manshans above.
\item[3.]
  Fight on, my brudder, for de manshans above,\\
  For I hope to meet my Jesus dere\\
  In de manshans above.
\end{stanza}


\newpage
\section{Shout on, children.}
\thispagestyle{empty}

\begin{song}
  \lilypondfile[staffsize=18,line-width=356\pt]{080.ly}
\end{song}

\begin{stanza}
\item[2.]
  Shout an' pray both night an' day;\\
  How can you die, you in de Lord?
\item[3.]
  Come on, chil'en, let's go home;\\
  O I'm so glad you're in de Lord.
\end{stanza}


\newpage
%% CHANGED: Changed to match table of content: bye to by.
\section{Jesus, won't you come by-and-by?}
\thispagestyle{empty}

\begin{song}
  \lilypondfile[staffsize=18,line-width=356\pt]{081.ly}
\end{song}


\newpage
\section{Heave away.}
\thispagestyle{empty}

\begin{song}
  \lilypondfile[staffsize=18,line-width=356\pt]{082.ly}
\end{song}

\begin{extra}
  [This is one of the Savannah firemen's songs of which Mr.~Kane
    O'Donnel gave a graphic acccount in a letter to the Philadelphia
    \emph{Press}.  ``Each company,'' he says, ``has its own set of
    tunes, its own leader, and doubtless in the growth of time,
    necessity and invention, its own composer.'']
\end{extra}



\part{Northern Seaboard Slave States, including Delaware, Maryland,
  Virginia, and North Carolina.}
\thispagestyle{empty}

\section{Wake up, Jacob.}
\thispagestyle{empty}

\begin{song}
  \lilypondfile[staffsize=18,line-width=356\pt]{083.ly}
\end{song}

\begin{stanza}
\item[2.]
  Got some friends on de oder shore,\\
  Do love de Lord!\\
  I want to see 'em more an' more,\\
  Do love de Lord!\\
  Wake up, Jacob, \&c.
\end{stanza}


\newpage
\section{On to Glory.}
\thispagestyle{empty}

\begin{song}
  \lilypondfile[staffsize=18,line-width=356\pt]{084.ly}
\end{song}

\begin{stanza}
\item[2.]
  Oh, there's Bill Thomas, I know him well,\\
  He's got to work to keep from hell;\\
  He's got to pray by night and day,\\
  If he wants to go by the narrow way.
\item[3.]
  There's Chloe Williams, she makes me mad,\\
  For you see I know she's going on bad;\\
  She told me a lie this arternoon,\\
  And the devil will get her very soon.
\end{stanza}

\begin{extra}
  [We should be tempted, from the character of this tune, to doubt its
    genuineness as a pure negro song.  We're informed, however, that
    it was sung twenty-five years ago in negro camp-meetings, and not
    in those of the whites.  The words, at any rate, are worth
    preserving, as illustrating the kind of influence brought to bear
    upon the wavering.]
\end{extra}


\newpage
\section{Just Now.}
\thispagestyle{empty}

\begin{song}
  \lilypondfile[staffsize=18,line-width=356\pt]{085.ly}
\end{song}

\begin{stanza}
\item[2.]
  Good religion, good religion, etc.
\item[3.]
  Come to Jesus, come to Jesus, etc.
\end{stanza}

\begin{extra}
  [This, which is now, in a somewhat different form, a Methodist hymn,
    was sung as given above, by the colored people of Ann Arundel Co.,
    Md., twenty-five years ago.---W.~A.~H.]
\end{extra}


\newpage
\section{Shock along, John.}
\thispagestyle{empty}

\begin{song}
  \lilypondfile[staffsize=18,line-width=356\pt]{086.ly}
\end{song}

\bigskip
\begin{extra}
  [A corn-song, of which only the burden is remembered.]
\end{extra}


\newpage
\section{Round the corn, Sally.}
\thispagestyle{empty}

\begin{song}
  \lilypondfile[staffsize=18,line-width=356\pt]{087.ly}
\end{song}

\begin{stanza}
\item[2.]
  Here's your iggle-quarter and here's your count-aquils.

\item[3.]
  I can bank, 'ginny bank, 'ginny bank the weaver.
\end{stanza}

\begin{extra}
  [``Iggle'' is of course ``eagle;'' for the rest of the enigmatical
    words and expressions in this corn-song, we must leave readers to
    guess at the interpretation.]
\end{extra}


\newpage
\section{Jordan's Mills.}
\thispagestyle{empty}

\begin{song}
  \lilypondfile[staffsize=18,line-width=356\pt]{088.ly}
\end{song}

\begin{stanza}
\item[2.]
  Built without nail or hammer.
\item[3.]
  Runs without wind or water.
\end{stanza}


\newpage
\section{Sabbath has no end.}
\thispagestyle{empty}

\begin{song}
  \lilypondfile[staffsize=18,line-width=356\pt]{089.ly}
\end{song}

\begin{stanza}
\item[2.]
  Gwine to follow King Jesus, I really do believe.
\item[3.]
  I love God certain.
\item[4.]
  My sister's got religion.
\item[5.]
  Set down in the kingdom.
\item[6.]
  Religion is a fortune.
\end{stanza}

\begin{extra}
  [This chorus was written down as exactly as possible from the lips
    of the singer, and illustrates the odd transformations which words
    undergo in their mouths.  It is a verse of a familiar hymn:
    ``fore-half'' is ``forehead;'' ``harpess'' is ``harp.'']
\end{extra}


\newpage
\section{I don't feel weary.}
\thispagestyle{empty}

\begin{song}
  \lilypondfile[staffsize=18,line-width=356\pt]{090.ly}
\end{song}

\begin{stanza}
\item[2.]
  Gwine to live with God forever, While, etc.
\item[3.]
  And keep the ark a-moving, While, etc.
\end{stanza}


\newpage
\section{The Hypocrite and the Concubine.}
\thispagestyle{empty}

\begin{song}
  \lilypondfile[staffsize=18,line-width=356\pt]{091.ly}
\end{song}


\newpage
\section{O shout away.}
\thispagestyle{empty}

\begin{song}
  \lilypondfile[staffsize=18,line-width=356\pt]{092.ly}
\end{song}

\begin{stanza}
\item[2.]
  O Satan told me not to pray,\\
  He want my soul at judgement day.
\item[3.]
  And every where I went to pray,\\
  There some thing was in my way.
\end{stanza}


\newpage
\section{O'er the Crossing.}
\thispagestyle{empty}

\begin{song}
  \lilypondfile[staffsize=18,line-width=356\pt]{093.ly}
\end{song}

\begin{stanza}
\item[2.]
  O yonder's my ole mudder, Been a waggin' at de hill so long;\\
  It's about time she cross over, Git home bime-by.\\
  Keep prayin', I do believe, etc.
\item[3.]
  O hear dat lumberin' thunder A-roll from do' to do',\\
  A-callin' de people home to God; Dey'll git home bime-by.\\
  Little chil'n, I do believe, etc.
\item[4.]
  O see dat forked lightnin' A-jump from cloud to cloud,\\
  A-pickin' up God's chil'n; Dey'll git home bime-by.\\
  Pray mourner, I do believe, etc.
\end{stanza}

\newpage
\thispagestyle{empty}

\begin{extra}
  [This ``infinitely quaint description of the length of the heavenly
    road,'' as Col.~Higginson styles it, is one of the most peculiar
    and wide-spread of the spirituals.  It was sung as given above in
    Caroline
    %% CHANGED: ``State.'' -> ``State,''
    Co., Virtginia, and probably spread southeard from this State,
    variously modified in different localities.  ``My body rock `long
    fever,'' (No.~45) would hardly be recognised as the same, either
    by words or tune, and yet it is almost certainly the same, as is
    shown by the following, sung in Augusta, Georgia, which has some
    of the words of the present song, adapted to a tune which is
    unmistakably identical with No.~45.]
\end{extra}

\bigskip
\begin{song}
  \begin[staffsize=18,line-width=356\pt]{lilypond}
\score
{
  \new Staff {
    <<
    \context Voice
    {
	\set Staff.midiInstrument = "acoustic grand"
     \override Staff.VerticalAxisGroup #'minimum-Y-extent = #'(0 . 0)
	
	\autoBeamOff

	\time 4/4
	\clef violin
	\key b \minor

	\partial 4 fis'4 | b'4 b' b' d''8[ b'] | d''4 d'' r fis'8 fis' |
	\break
	a'8 a' a' a' b'4 a' | fis'2 r4 fis' |
	\break
	b'8 b' b' b' b'4 d''8 b' | d''4 d'' d'' r8 d'' |
	\break
	d''4 d''8 d'' e''[ d''] b''[ a''] | b'2 r4
	\bar "||"
    }
    \lyricsto "" \new Lyrics
    {
      \override LyricText #'font-size = #0
      \override StanzaNumber #'font-size = #-1

      O yon -- der's my ole moth -- er,
      Been a -- wag -- gin' at de hill so long;
      I real -- ly do be -- lieve she's a child of God,
      She'll git home to heav'n bime -- bye.
    }
    >>
  }

  \layout { indent = 0.0 }
}
\end{lilypond}
\end{song}

\bigskip
\begin{extra}
  [We regret we have not the air of the Nashville variation, ``My Lord
    called Daniel.'']
\end{extra}


\newpage
\section{Rock o' my Soul.}
\thispagestyle{empty}

\begin{song}
  \lilypondfile[staffsize=18,line-width=356\pt]{094.ly}
\end{song}


\newpage
%% CHANGED: Title in book has ``through''.
\section{We will march thro' the valley.}
\thispagestyle{empty}

\begin{song}
  \lilypondfile[staffsize=18,line-width=356\pt]{095.ly}
\end{song}

\begin{stanza}
\item[2.]
  We will march, etc.\\
  Behold I give myself away, and\\
  We will march, etc.
\item[3.]
  We will march, etc.\\
  This track I'll see and I'll pursue;\\
  We will march, etc.
\item[4.]
  We will march, etc.\\
  When I'm dead and buried in the cold silent tomb,\\
  I don't want you to grieve for me.
\end{stanza}


\newpage
\section{What a trying time.}
\thispagestyle{empty}

\begin{song}
  \lilypondfile[staffsize=18,line-width=356\pt]{096.ly}
\end{song}

\begin{stanza}
\item[2.]
  Lord, I am in the garden.
\item[3.]
  Adam, you ate that apple.
\item[4.]
  Lord, Eve she gave it to me.
\item[5.]
  Adam, it was forbidden.
\item[6.]
  Lord, said, walk out de garden.
\end{stanza}

\begin{extra}
  [A most compendious account of the fall.]
\end{extra}

\newpage
\section{Almost Over.}
\thispagestyle{empty}

\begin{song}
  \lilypondfile[staffsize=18,line-width=356\pt]{097.ly}
\end{song}

\begin{stanza}
\item[2.]
  Sister, if your heart is warm,\\
  Snow and ice will do you no harm.
\item[3.]
  I done been down, and I done been tried,\\
  I been through the water, and I been baptized.
\item[4.]
  O sister, you must mind how you step on the cross,\\
  Your foot might slip, and your soul get lost.
\item[5.]
  And when you get to heaven, you'll be able for to tell\\
  How you shunned the gates of hell.
\item[6.]
  Wrestle with Satan and wrestle with sin,\\
  Stepped over hell and come back again.
\end{stanza}

\begin{extra}
  [A baptismal song, as the chattering ``almost o-ver'' so forcibly
    suggests.]
\end{extra}


\newpage
\section{Don't be weary, traveller.}
\thispagestyle{empty}

\begin{song}
  \lilypondfile[staffsize=18,line-width=356\pt]{098.ly}
\end{song}

\begin{stanza}
\item[2.]
  Where to go I did not know\\
  Ever since he freed my soul.
\item[3.]
  I look at de worl' and de worl' look new,\\
  I look at de worl' and de worl' look new.
\end{stanza}


\newpage
\section{Let God's saints come in.}
\thispagestyle{empty}

\begin{song}
  \lilypondfile[staffsize=18,line-width=356\pt]{099.ly}
\end{song}

\begin{stanza}
\item[2.]
  There was a wicked man,\\
  He kept them children in Egypt Land.
\item[3.]
  God did say to Moses one day,\\
  Say, Moses go to Egypt land,
\item[4.]
  And tell him to let my people go.\\
  And Pharaoh would not let 'em go.
\item[5.]
  God did go to Moses' house,\\
  And God did tell him who he was,
\item[6.]
  God and Moses walked and talked,\\
  And God did show him who he was.
\end{stanza}


\newpage
\section{The Golden Altar.}
\thispagestyle{empty}

\begin{song}
  \lilypondfile[staffsize=18,line-width=356\pt]{100.ly}
\end{song}

\begin{stanza}
\item[2.]
  And home to Jesus we will go, we will go, etc.;\\
  We are de people of de Lord.
  John sawr-O, etc.
\item[3.]
  Dere's a golden slipper in de heaven for you, etc.,\\
  Before de Lamb of God.
\item[4.]
  I wish I'd been dere when prayer begun, etc.
\item[5.]
  To see my Jesus about my sins, etc.
\item[6.]
  Then home to glory we will go, etc.
\end{stanza}

\begin{extra}
  [This is interesting as an undoubted variation of ``John, John of the
    holy order.'' No.~22.  A comparison of the words shows that the word
    ``number'' should be ``member.'']
\end{extra}


\newpage
\section{The Winter.}
\thispagestyle{empty}

\begin{song}
  \lilypondfile[staffsize=18,line-width=356\pt]{101.ly}

  \footnotetext[1]{Am a-comin'.}
  \footnotetext[2]{Sing.}
\end{song}

\begin{stanza}
\item[2.]
  You bend your knees\footnote[3]{I bend my knees, etc.} on holy ground, ground,\\
  And ask de Lord, Lord, for to turn you around.
  For de vinter, etc.
\item[3.]
  I turn my eyes towards the sky, sky,\\
  And ask de Lord, Lord, for wings to fly.
\item[4.]
  For you see me gwine 'long so, so,\\
  I has my tri-trials yer below.
\end{stanza}


\newpage
\section{The Heaven Bells.}
\thispagestyle{empty}

\begin{song}
  \lilypondfile[staffsize=18,line-width=356\pt]{102.ly}
\end{song}


\newpage
\thispagestyle{empty}


\part{Inland Slave States, including Tennessee, Arkansas,
  and the Mississippi River.}
\thispagestyle{empty}


\newpage
\section{The Gold Band.}
\thispagestyle{empty}

\begin{song}
  \lilypondfile[staffsize=18,line-width=356\pt]{103.ly}
\end{song}

\begin{stanza}
\item[2.]
  Sister Mary gwine to hand down the robe,\\
  In the army, bye-and-bye;\\
  Gwine to hand down the robe and the gold band,\\
  In the army, bye-and-bye.
\end{stanza}


\newpage
\section{The Good Old Way.}
\thispagestyle{empty}

\begin{song}
  \lilypondfile[staffsize=18,line-width=356\pt]{104.ly}

  \footnotetext[1]{Sister, etc.}
\end{song}
	
\newpage
\section{I'm going home.}
\thispagestyle{empty}

\begin{song}
  \lilypondfile[staffsize=18,line-width=356\pt]{105.ly}
\end{song}

\begin{stanza}
\item[2.]
  I found free grace in the wilderness.
\item[3.]
  My father preaches in the wilderness.
\end{stanza}


\newpage
\section{Sinner won't die no more.}
\thispagestyle{empty}

\begin{song}
  \lilypondfile[staffsize=18,line-width=356\pt]{106.ly}
\end{song}

\begin{stanza}
\item[2.]
  O see dem ships come a-sailing, sailing, sailing,\\
  O see dem ships come a-sailing,\\
  De robes all ready now.
\end{stanza}


\newpage
\section{Brother, guide me home.}
\thispagestyle{empty}

\begin{song}
  \lilypondfile[staffsize=18,line-width=356\pt]{107.ly}
\end{song}

\begin{stanza}
\item[2.]
  Let's go to God, chil'n, (\emph{ter})\\
  Bright angels biddy me to come.
\end{stanza}

\begin{extra}
  [I heard this in a praise-house at the ``Contraband Camp'' on
    President's Island near Memphis, in September, 1864.  I will not vouch
    for the absolute accuracy of my memory.---W.~F.~A.]
\end{extra}


\newpage
\section{Little children, then won't you be glad?}
\thispagestyle{empty}

\begin{song}
  \lilypondfile[staffsize=18,line-width=356\pt]{108.ly}
\end{song}

\begin{stanza}
\item[2.]
  King Jesus, he was so strong (\emph{ter}), my Lord,\\
  That he jarred down the walls of hell.
\item[3.]
  Don't you hear what de chariot any? (\emph{bis})\\
  De fore wheels run by de grace ob God,\\
  An 'de hind wheels dey run by faith.
\item[4.]
  Don't you 'member what you promise de Lord? (\emph{bis})\\
  You promise de Lord that you would feed his sheep,\\
  An' gather his lambs so well.
\end{stanza}

\begin{extra}
  [Often sung in the colored schools at Helena, Arkansas.]
\end{extra}


\newpage
\section{Charleston Gals.}
\thispagestyle{empty}

\begin{song}
  \lilypondfile[staffsize=18,line-width=356\pt]{109.ly}
\end{song}

\begin{stanza}
\item[2.]
  As I went a-walking down the street,\\
  Up steps Charleston gals to take a walk with me.\\
  I kep' a walking and they kep' a talking,\\
  I danced with a gal with a hole in her stocking.
\end{stanza}


\newpage
\section{Run, nigger, run.}
\thispagestyle{empty}

\begin{song}
  \lilypondfile[staffsize=18,line-width=356\pt]{110.ly}
\end{song}


\newpage
\section{I'm gwine to Alabamy.}
\thispagestyle{empty}

\begin{song}
  \lilypondfile[staffsize=18,line-width=356\pt]{111.ly}
\end{song}

\begin{stanza}
\item[2.]
  She went from Old Virginny,---Oh, \\
  And I'm her pickaninny,---Ah.

\item[3.]
  She lives on the Tombigbee,---Oh, \\
  I wish I had her wid me,---Ah.

\item[4.]
  Now I'm a good big nigger,---Oh, \\
  I reckon I won't get bigger,---Ah.

\item[5.]
  But I'd like to see my mammy,---Oh, \\
  Who lives in Alabamy,---Ah.
\end{stanza}


\newpage
\thispagestyle{empty}


\part{Gulf States, including Florida and Louisiana: Miscellaneous}
\thispagestyle{empty}


\newpage
\section{My Father, how long?}
\thispagestyle{empty}

\begin{song}
  \lilypondfile[staffsize=18,line-width=356\pt]{112.ly}
\end{song}
\footnotetext[1]{Mother, etc.}

\begin{stanza}
\item[2.]
  We'll soon be free, (\emph{ter})\\
  De Lord will call us home.
\item[3.]
  We'll walk de miry road\\
  Where pleasure never dies.
\item[4.]
  We'll walk de golden streets\\
  Of de New Jerusalem.
\item[5.]
  My brudders do sing\\
  De praise of de Lord.
\item[6.]
  We'll fight for liberty\\
  When de Lord will call us home.
\end{stanza}

\begin{extra}
  [For singing this ``the negroes had been put in jail at Georgetown,
    S.~C., at te outbreak of the Rebellion.  `We'll soon be free' was
    too dangerous an assertion, and though the chant was an old one,
    it was no doubt sung with re-doubled emphasis during the new
    events.  `De Lord will call us home,' was evidently thought to be
    a symbolical verse; for, as a little drummer boy explained it to
    me, showing all his white teeth as he sat in the moonlight by the
    door of my tent, `Dey tink \emph{de Lord} mean for say \emph{de
      Yankees}.'{}''---T.~W.~H.]
\end{extra}

\newpage
\section{I'm in trouble}
\thispagestyle{empty}

\begin{song}
  \lilypondfile[staffsize=18,line-width=356\pt]{113.ly}
\end{song}


\newpage
\section{O Daniel.}
\thispagestyle{empty}

\begin{song}
  \lilypondfile[staffsize=18,line-width=356\pt]{114.ly}
\end{song}


\newpage
\section{O brother, don't get weary.}
\thispagestyle{empty}

\begin{song}
  \lilypondfile[staffsize=18,line-width=356\pt]{115.ly}
\end{song}


\newpage
\section{I want to join the band.}
\thispagestyle{empty}

\begin{song}
  \lilypondfile[staffsize=18,line-width=356\pt]{116.ly}
\end{song}


\newpage
\section{Jacob's Ladder.}
\thispagestyle{empty}

\begin{song}
  \lilypondfile[staffsize=18,line-width=356\pt]{117.ly}
\end{song}


\newpage
\section{Pray on.}
\thispagestyle{empty}

\begin{song}
  \lilypondfile[staffsize=18,line-width=356\pt]{118.ly}
\end{song}

\bigskip
\begin{extra}
  [As an interpretation of ``dem light us over,'' I suggest ``de night
    is over;'' and ``union'' should probably have a capital \emph{U}.
    ``De night is over; de Union break of day (da comin').''  The
    interchange of \emph{l} and \emph{n} is not uncommon, and is
    illustrated again in this song in the word ``Union,'' which was
    pronounced ``yuliul'' by the person who sang it to me.  This song
    and Nos.~38, 41, 42, 43, 119, 122, and 123, came on to the
    plantation after I left.---C.~.P.~W.]
\end{extra}


\newpage
\section{Good news, Member.}
\thispagestyle{empty}

\begin{song}
  \lilypondfile[staffsize=18,line-width=356\pt]{119.ly}
\end{song}

\begin{stanza}
\item[2.]
  Mr.~Hawley have a home in Paradise.
\item[3.]
  Archangel bring baptizing down.
\end{stanza}


\newpage
\section{I want to die like-a Lazarus die.}
\thispagestyle{empty}

\begin{song}
  \lilypondfile[staffsize=18,line-width=356\pt]{120.ly}
\end{song}


\newpage
\section{Away down in Sunbury.}
\thispagestyle{empty}

\begin{song}
  \lilypondfile[staffsize=18,line-width=356\pt]{121.ly}
\end{song}


\newpage
\section{This is the trouble of the world.}
\thispagestyle{empty}

\begin{song}
  \lilypondfile[staffsize=18,line-width=356\pt]{122.ly}
\end{song}
\footnotetext[1]{(What you doubt for?) etc.}
\footnotetext[2]{(what you shame for?), (take it easy), (Titty Melia)}


\newpage
\section{Lean on the Lord's side.}
\thispagestyle{empty}

\begin{song}
  \lilypondfile[staffsize=18,line-width=356\pt]{123.ly}
\end{song}
\footnotetext[1]{\emph{i.~e.} Daniel (as if Samson) racked the lion's
  \emph{jaw.}}
\footnotetext[2]{Band.}

\bigskip
\begin{extra}
  A Port Royal variation of ``Who is on the Lord's side'' (No.~75.)
\end{extra}


\newpage
\section{These are all my Father's children.}
\thispagestyle{empty}

\begin{song}
  \lilypondfile[staffsize=18,line-width=356\pt]{124.ly}
\end{song}

\bigskip
\begin{extra}
  [This is interesting as being probably the original of ``Trouble of
    the world'' (No.~10,) and peculiarly so from the following custom,
    which is described by a North Carolina negro as existing in South
    Carolina.  When a \emph{pater-familias} dies, his family assemble
    in the room where the coffin is, and, ranging themselves round the
    body in the order of age and relationship, sing this hymn,
    marching round and round.  They also take the youngest and pass
    him first over and then under the coffin.  Then two men take the
    coffin on their shoulders and carry it on the run to the grave.]
\end{extra}


\newpage
\section{The Old Ship of Zion.}
\thispagestyle{empty}

\begin{extra}
  [We have received two versions of the ``Old Ship of Zion,'' quite
    different from each other and from those given by Col.~Higginson.
    The first was sung twenty-five years ago by the colored people of
    Ann Arundel Co., Maryland.  The words may be found in ``The
    Chorus'' (Philadelphia: A.~S.~Jenks, 1860,) p.~170.  (Compare,
    also, p.~167.)
\end{extra}

\bigskip
\begin{song}
  \lilypondfile[staffsize=18,line-width=356\pt]{125a.ly}
\end{song}

\begin{stanza}
\item[2.]
  And who is the Captain of the ship that you're on?---O glory, etc.\\
  My Saviour is the Captain, hallelujah!
\end{stanza}


\newpage
\thispagestyle{empty}

\begin{extra}
  [The other is from North Carolina:]
\end{extra}

\bigskip
\begin{song}
  \lilypondfile[staffsize=18,line-width=356\pt]{125b.ly}
\end{song}

\begin{stanza}
\item[2.]
  She sails like she is heavy loaded.
\item[3.]
  King Jesus is the Captain.
\item[4.]
  The Holy Ghost is the Pilot.
\end{stanza}


\newpage
\section{Come along, Moses.}
\thispagestyle{empty}

\begin{song}
  \lilypondfile[staffsize=18,line-width=356\pt]{126.ly}
\end{song}
\footnotetext[1]{Judy, Aaron.}
\footnotetext[2]{Children.}

\begin{stanza}
\item[2.]
  He sits in the Heaven and he answers prayer.
\item[3.]
  Stretch out your rod and come across.
\end{stanza}

\begin{extra}
  [This air has in parts a suspicious resemblance to the Sunday-school
    hymn ``{}'Tis religion that can give,'' which has become very
    wide-spread in the South since the war.  Mrs.~James, however,
    heard it from an old woman in North Carolina, early in 1862, which
    would seem to vouch for its genuineness.]
\end{extra}


\newpage
\section{The Social Band.}
\thispagestyle{empty}

\begin{song}
  \lilypondfile[staffsize=18,line-width=356\pt]{127.ly}
\end{song}
\footnotetext[1]{Brother David.}


\newpage
\section{God got plenty o' room.}
\thispagestyle{empty}

\begin{song}
  \lilypondfile[staffsize=18,line-width=356\pt]{128.ly}
\end{song}

\begin{stanza}
\setlength{\itemsep}{1pt}
\setlength{\parskip}{0pt}
\item[2.]
  So many-a weeks and days have passed\\
  Since we met together last.
\item[3.]
  Old Satan tremble when he sees\\
  The weakest saints upon their knees.
\item[4.]
  Prayer makes the darkest cloud withdraw,\\
  Prayer climbed the ladder Jacob saw.
\item[5.]
  Daniel's wisdom may I know,\\
  Stephen's faith and spirit sure.
\item[6.]
  John's divine communion feel,\\
  Joseph's meek and Joshua's zeal.
\item[7.]
  There is a school on earth begun\\
  Supported by the Holy One.
\item[8.]
  We soon shall lay our school-books by,\\
  And shout salvation as I fly.
\end{stanza}

\begin{extra}
  [The above is given exactly as it was sung, some of the measures in
    %% FIXME: Put in real measure signs 2/8 etc...
    2/8, some in 3/8, and some in 2/4 time.  The irregularity probably
    arises from omission of rests, but it seemed a hopeless undertaking
    to attempt to restore the correct time, and it was thought best to
    give it in this shape as at any rate a characteristic specimen of
    negro singing.  The song was obtained of a North Carolina negro, who
    said it came from Virginia.]
\end{extra}

\newpage
\section{You must be pure and holy.}
\thispagestyle{empty}

\begin{song}
  \lilypondfile[staffsize=18,line-width=356\pt]{129.ly}
\end{song}

\begin{stanza}
\item[2.]
  I'll run all round the cross and cry,\\
  My Lord, bretheren, ah my Lord,\\
  Or give me Jesus, or I die,\\
  My Lord, bretheren, ah my Lord.\\
  You must be pure and holy, etc.
\item[3.]
  The Devil am a liar and conjurer too, My Lord, etc.\\
%% FIXME: cut you in two, cut you through
  If you don't look out he'll conjure you, My Lord, etc.
\item[4.]
  O run up, sonny, and get your crown, My Lord, etc.\\
  And by your Father sit you down, My Lord, etc.
\item[5.]
  I was pretty young when I begun, My Lord, etc.\\
  But, now my work is almost done, My Lord, etc.
\item[6.]
  The Devil's mad and I am glad, My Lord, etc.\\
  He lost this soul, he thought he had, My Lord, etc.
\item[7.]
  Go 'way, Satan, I don't mind you, My Lord, etc.\\
  You wonder, too, that you can't go through, My Lord, etc.
\item[8.]
  A lilly\footnote[1]{\emph{Qu.} little. ?} white stone came rolling down,
  My Lord, etc.\\
  It rolled like thunder through the town, My Lord, etc.
\end{stanza}

\begin{extra}
  [This is a favorite and apparently genuine song which ``flourishes''
    in a colored church at Auburn, N.~.Y. having been introduced there
    from the South.  ``It is sung on \emph{all} occasions, and without
    any regard to \emph{order} in the verses; you may not be able to
    see any connection between any of them.  The horus is always sung
    once ot twice before the verses are used at all.  You will see
    that occasionally there is inserted an extra sylable (ah) and
    always in the 2nd and 4th lines of the verses; why this is done I
    am unable to discover, but it appears to assist them wonderfully
    in singing.  The first note in the chorus is sung very
    \emph{loud,} and is prolonged to an indefinite time, at the
    pleasure of the leader.  You will notice that the air is in the
    minor mode, but the chorus, with the exception of the last line,
    in the major.''---W.~A.~B.]
\end{extra}


\newpage
\section{Belle Layotte.}
\thispagestyle{empty}

\begin{song}
  \lilypondfile[staffsize=18,line-width=356\pt]{130.ly}
\end{song}

\begin{stanza}
\item[2.]
  Jean Babet, mon ami,\\ 
  Si vous couri par en haut,\\
  Vous mand\'e belle Layotte\\
  Cadeau la li t\'e promi mouin.
\item[3.]
  Domestique la maison\\
  Y\'e tout fach\'e avec mouin,\\
  Paraporte chanson la\\
  Mo compos\'e pou la belle Layotte.
\end{stanza}


\newpage
\section{Remon.}
\thispagestyle{empty}

\begin{song}
  \lilypondfile[staffsize=18,line-width=356\pt]{131.ly}
\end{song}


\newpage
\section{Aurore Bradaire.}
\thispagestyle{empty}

\begin{song}
  \lilypondfile[staffsize=18,line-width=356\pt]{132.ly}
\end{song}


\newpage
\section{Caroline.}
\thispagestyle{empty}

\begin{song}
  \lilypondfile[staffsize=18,line-width=356\pt]{133.ly}
\end{song}


\newpage
\section{Calinda.}
\thispagestyle{empty}

\begin{song}
  \lilypondfile[staffsize=18,line-width=356\pt]{134.ly}
\end{song}

\begin{stanza}
\item[2.]
  Michi\'e Pr\'eval li t\'e capitaine bal,\\
  So cocher Louis t\'e maite c\'er\'emonie.
\item[3.]
  Dans lequirie la yav\'e gran gala,\\
  Mo cr\'e choual lay\'e t\'e bien \'etonn\'e.
\item[4.]
  Yav\'e des n\'egresse belle pass\'e maitresse,\\
  Y\'e vol\'e b\'ebelle dans l'ormoire mamzelle.
\end{stanza}


\newpage
\section{Lolotte.}
\thispagestyle{empty}

\begin{song}
  \lilypondfile[staffsize=18,line-width=356\pt]{135.ly}
\end{song}


\newpage
\section{Musieu Bainjo.}
\thispagestyle{empty}

\begin{songpart}
  \lilypondfile[staffsize=18,line-width=356\pt]{136-p1.ly}
\end{songpart}
\begin{song}
  \lilypondfile[staffsize=18,line-width=356\pt]{136-p2.ly}
\end{song}

\newpage
\thispagestyle{empty}

\begin{extra}
  [The seven foregoing songs were obtained from a lady who heard them
    sung, before the war, on the ``Good Hope'' plantation, St.~Charles
    Parish, Louisiana.  The language, evidently a rude corruption of
    French, is that spoken by the negroes in that part of the State;
    and it is said that it is more difficult for persons who speak
    French to interpret this dialect, than for those who speak English
    to understand the most corrupt of the ordinary negro-talk.  The
    pronunciation of this negro-French is indicated, as accurately as
    possible, in the versions given here, which furnish, also, many
    interesting examples of the peculiar phrases and idioms employed
    by this people.  The frequent omission of prepositions, articles,
    and auxiliary verbs, as well as of single letters, and the
    contractions constantly occurring, are among the most noticeable
    peculiarities.  Some of the most difficult words are: \emph{mo
      \emph{for} me, mon, je; li \emph{for} lui, le, la il, elle;
      mouin \emph{for} moi; y \emph{for} ils, leur; aine, d \emph{for}
      un, deux; t \emph{for} t, tait; ya, yav \emph{for} il y a,
      \emph{etc.;} ouar \emph{for} voir \emph{and its inflections;}
      oul \emph{for} vouloir, \emph{etc.;} pancor \emph{for} pas
      encore; michi \emph{for} monsieur; inp \emph{for} un peu.}  The
    words are, of course, to be pronounced as if they were pure
    French.

    Four of these songs, Nos.~130, 131, 132 and 133, were sung to a
    simple dance, a sort of minuet called the \emph{Coonjai}; the name
    and the dance are probably both of African origin.  When the
    \emph{Coonjai} is danced, the music is furnished by an orchestra
    of singers, the leader of whom---a man selected both for the
    quality of his voice and for his skill in improvising---sustains
    the solo part, while the others afford him an opportunity, as they
    shout in chorus, for inventing some neat verse to compliment some
    lovely \emph{danseuse}, or celebrate the deeds of some plantation
    hero.  The dancers themselves never sing, as in the case of the
    religious ``shout'' of the Port Royal negroes; and the usual
    musical accompaniment, besides that of the singers, is that
    furnished by a skilful performer on the barrel-head-drum, the
    jaw-bone and key, or some other rude instrument.

    No.~134.  The ``calinda'' was a sort of contra-dance, which has
    now passed entirely out of use.  Bescherelle describes the two
    lines as ``avan\c{}cant et reculant en cadence, et faisant des
    contorsions for singuli\'eres et des gestes fort lascifs.''

    The first movement of No.~135, ``Lolotte,'' has furnished
    M.~Gottschalk with the theme of his ``Ballade Cr\'eole,'' ``La
    Savane,'' op.~3 de la Louisiane.

    In 136, we have the attempt of some enterprising negro to write a
    French song; he is certainly to be congratulated on his success.

    It will be noticed that all these songs are ``seculars''; and that
    while the words of most of them are of very little account, the
    music is as peculiar, as interesting, and, in the case of two or
    three of them, as difficult to write down, or to sing correctly,
    as any that have preceded them.]
\end{extra}


\newpage
%% \changepage{textheight}{textwidth}{evensidemargin}{oddsidemargin}{columnsep}{topmargin}{headheight}{headsep}{footskip}
\changepage{-180pt}{-80pt}{+65pt}{+15pt}{}{+90pt}{}{}{}
\setlength{\parindent}{\parindentsave}

\pagestyle{myheadings}
\markboth{\uppercase{Editors'\ Note}}{}

\chapter*{Editors' Note}

The original arrangement of the foregoing collection has not been
adhered to.  Why the secular songs do not appear by themselves has
been already explained.  That the division into parts is not strictly
geographical was caused by the tardy arrival of most of the songs
contained in Part IV.  Should a second edition ever be justified by
the favor with which the present is received, these irregularities
will be corrected.

It was proposed to print music without words, and words without music,
each by themselves.  But the first can hardly be said to have been
obtained, unless ``Shock along, John,'' No.~86, is an instance.  The
words without music which in one or two cases were kindly, and we fear
laboriously, communicated to us, presented no fresh or striking
peculiarities, and we therefore decided against their admission.

As was remarked in the Introduction, we are fully aware of the
incompleteness of this collection, though we may fairly enough assume
it to be \emph{la cr\^eme de la cr\^eme}.  Col.~Higginson writes:

``I wish you would look up one `spiritual,' of which I only remember
the chorus---`\emph{It doth appear}'---as being often sung in camp.
Also, `\emph{Ring dat charming bell},' which they used to sing to
please Mrs.~Saxton, who liked it.''

Gen.~James H.~Wilson, who, in the earlier part of the war, was at Port
Royal, and, during explorations and night surveys of the coast between
there and Ossabaw Sound, had frequent opportunities of hearing every
grade of ``spirituals,'' writes, of Col.~Higginson's collection in the
Atlantic Monthly:

``He has omitted two which I heard more generally sung than any
others.  I refer to the one beginning:

``{}`They took ole Master Lord,\\
And fed him on pepper and gall,'

and the other:

``{}`My brudder Johnny's new-born baby,\\
Hi oh, de new-born!'

``The airs to which these were sung are very peculiar, while the
burthen of the songs is pretty clearly indicated by the lines given
above.  The first seems to allude to the persecutions of Christ, while
the latter simply refers to the birth and early death of a new-born
baby, and is varied by making a new verse for all the brothers and
sisters that the singer happens to be able to call to mind.

``I also recollect the refrain of a boat-song which a crew of ten
stalwart negroes used to sing for me in our excursions, but I am
inclined to the belief that it was by no means a `spiritual,' as I
could never get any of them to explain it to my satisfaction.  The
only words I could make out clearly were:

``{}`Jah de window, jah!\\
Oh jah de window,' etc.

and the meaning of which I took to be `{}'jar the window.'  If they
had any such thing among them, this was probably a fragment of a
simple ballad, describing an incident in a negro courtship.  I got
this impression at the time, partly from the peculiar tone of the
song, and partly from their hesitancy to explain it.  But whatever may
have been its real character, it was quite musical, and had such an
inspiring effect upon my boatmen that I have known them to row
eighteen or twenty miles, exerting their utmost strength, keeping
perfect time to its notes, and never pausing for breath.''

These, certainly, are songs to be desired and regretted.  But we do
not despair of recovering them and others perhaps equally
characteristic for a second edition; and we herewith solicit the kind
offices of collectors into whose hands this volume may have fallen, in
extending and perfecting our researches.  For fully a third of the
songs recorded by Col.~Higginson we have failed to obtain the music,
and they may very well serve as a guide for future investigators.  We
shall also gratefully acknowledge any errors of fact or of typography
that may be brought to our attention, and in general anything that
would enhance the value or the interest of this collection.
Communications may be addressed to Mr.~W.~P.~Garrison, Office of
\emph{The Nation} newspaper, New York City.

\textsc{November}, 1867. 

\end{document}
